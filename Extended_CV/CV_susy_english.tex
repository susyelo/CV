	%% start of file `template_en.tex'.
%% Copyright 2007 Xavier Danaux (xdanaux@gmail.com).
%
% This work may be distributed and/or modified under the
% conditions of the LaTeX Project Public License version 1.3c,
% available at http://www.latex-project.org/lppl/.


\documentclass[11pt,a4paper]{moderncv}
\usepackage[latin1]{inputenc}
\usepackage{url}



% moderncv themes
%\moderncvtheme[blue]{casual}                 % optional argument are 'blue' (default), 'orange', 'red', 'green', 'grey' and 'roman' (for roman fonts, instead of sans serif fonts)
\moderncvtheme[green]{classic}                % idem

% character encoding
%\usepackage[utf8]{inputenc}                   % replace by the encoding you are using

% adjust the page margins
\usepackage[scale=0.8]{geometry}
\recomputelengths                             % required when changes are made to page layout lengths

% personal data
\firstname{\huge Susy}
\small \familyname{\huge Echeverr\'ia-Londo\~no, PhD. }
\title{NSF Postdoctoral Associate}               % optional, remove the line if not wanted
\address{\hspace{-1in}Kenyon College}{Gambier OH, USA}{}    % optional, remove the line if not wanted
\mobile{+1(740)5046838}                    % optional, remove the line if not wanted
%\phone{+5776348129}                      % optional, remove the line if not wanted
%\fax{312 996 1491}
\social[github]{susyelo}                      % optional, remove the line if not wanted
\email{susyelo@gmail.com}
                    % optional, remove the line if not wanted
\extrainfo{echeverrialondono1@kenyon.edu} % optional, remove the line if not wanted
%\photo[64pt]{./cv.jpg}                         % '64pt' is the height the picture must be resized to and 'picture' is the name of the picture file; optional, remove the line if not wanted
%\quote{"No man should escape our universities without knowing how little he knows." -- J. Robert Oppenheimer}                 % optional, remove the line if not wanted

%\nopagenumbers{}                             % uncomment to suppress automatic page numbering for CVs longer than one page

\newcommand\myitem{\item[\textbullet]}

\newcommand{\cvreference}[7]{%
    \textbf{#1}\newline% Name
    \ifthenelse{\equal{#2}{}}{}{\addresssymbol~#2\newline}%
    \ifthenelse{\equal{#3}{}}{}{#3\newline}%
    \ifthenelse{\equal{#4}{}}{}{#4\newline}%
    \ifthenelse{\equal{#5}{}}{}{#5\newline}%
    \ifthenelse{\equal{#6}{}}{}{\emailsymbol~\texttt{#6}\newline}%
    \ifthenelse{\equal{#7}{}}{}{\phonesymbol~#7}}
%----------------------------------------------------------------------------------
%            content
%----------------------------------------------------------------------------------
\begin{document}
\maketitle

\section{SUMMARY}

I am an early career researcher focusing on the study of large-scale patterns of biodiversity and the ecological and evolutionary processes that shaped them. I am also interested in understanding how human pressures have affected and will affect the biodiversity in tropical regions. Throughout my career, I have gained extensive experience in the field of data science including data processing and visualisation, advanced statistical and spatial analysis. In my work, I particularly enjoy exploring and analysing biological data through visualisations tools.  

\section{EMPLOYMENT}

\cventry{2018-- }{Visitor scholar}{University of Pittsburgh}{\textit{Department of Biological Sciences}}{Pittsburgh, PA 15260, United States of America}{}
\cvline{}{PI: Prof. Justin Kitzes}
\cvline{}{Exploring signals of past spatial patterns to predict extinction risk of plants using time-series spatial point patterns from the Barro Colorado Island}

\cventry{2017-- }{NSF Postdoctoral Associate}{Kenyon College}{\textit{Department of Biology}}{Gambier, OH 43022, United States of America}{}
\cvline{}{PIs: Prof. Andrew Kerkhoff and Prof. Brian J Enquist}
\cvline{}{%
	\begin{itemize}
		\item Researching the consequences of climatic stability, biome age and area effects on seed plant diversity. 
		\item Carrying out paleoclimate reconstructions using machine learning methods such as random forest to calibrate biogeographic biomes inferences.
		\item Co-teaching of the BSc course ``Global Ecology and Biogeography'' at Kenyon College. 
		\item Taught and organized lectures and assignments for the Ecoinformatics skills course including Introduction to data science R, data managing and processing, data visualization, spatial analysis and reproducibility \url{https://globalecologybiogeography.github.io/Ecoinformatics/}		
\end{itemize}}
\cventry{2017}{Lecturer and demonstrator}{MSc Taxonomy and biodiversity, Methods in Macroecology and Macroevolution course, Natural History Museum (NHM) and Imperial College London }{London}{UK}{Phylogenetic approaches to studying diversification; Fossils in phylogenetics}

\cventry{2013--2016}{Demonstrator}{Computational Biostatistics BSc course, Ecology and Evolution BSc course, Biodiversity and Conservation Biology BSc course, Imperial College London}{London}{UK}{Introduction to R, Fundamentals of statistics in R, Phylogeny of mammals and pines, IUCN Red List, Biodiversity among lineages and over time, Delimiting species, Extinction risk patterns and correlates.}

%\cventry{2015}{Demonstrating}{Ecology and Evolution BSc course, Imperial College London}{London}{UK}{Phylogeny of mammals and pines}
%
%\cventry{2014}{Demonstrating}{Biodiversity and Conservation Biology BSc course, Imperial College London}{London}{UK}{IUCN Red List}
%
%\cventry{2013}{Demonstrating}{Biodiversity and Conservation Biology BSc course, Imperial College London}{London}{UK}{R basics, Biodiversity among lineages and over time practical, Delimiting species, Extinction risk patterns and correlates}

\cventry{2011-2012}{Science teacher}{New Cambridge School, Gimnasio pontevedra school, and La Quinta del puente school}{Bucaramanga}{Colombia}{}

%\cventry{2012}{Science teacher}{Gimnasio pontevedra school}{Bucaramanga}{Colombia}{}

%\cventry{2011}{Head of science department and teacher}{La Quinta del puente school}{Floridablanca}{Colombia}{}

\cventry{2008--2010}{Teaching Assistant and demonstrator}{Systematics BSc course, Universidad Industrial de Santander; Bioinformatics MSc course, Universidad Industrial de Santander}{Bucaramanga}{Colombia}{}

% arguments 3 to 6 are optional
%\cventry{2009}{Teaching Assistant}{Bioinformatics MSc course, Universidad Industrial de Santander}{Bucaramanga}{Colombia}{}
%Masters of Basic Biomedical Sciences Program                       % arguments 3 to 6 are optional
%\subsection{Miscellaneous}
%\cventry{year--year}{Job title}{Employer}{City}{}{Description line 1\newline{}Description line 2}% arguments 3 to 6 are optional


\section{EDUCATION}

\cventry{2013--2017}{PhD in Life Science}{Imperial College London}{\textit{Natural History Museum}}{London, United Kingdom}{}

\cvline{}{\textbf{Thesis:} Diversification patterns of \textit{Solanum} L. (Solanaceae), plant macroecology and responses to land-use change.}
\cvline{}{Supervisor: Prof. Andy Purvis; Co-supervisor Dr Sandra Knapp}
\cvline{}{%
	\begin{itemize}
		\item Researched the dynamics of diversification of the megadiverse plant genus Solanum using a collation of geographical, genetic and taxonomic data along with phylogenetic comparative methods. 
		\item Studied large-scale patterns of plant diversity and their responses to land-use change from data of a global collation of local field-based studies. 
		\item Carried out large-scale evolutionary modelling exercise on a high-performance Linux cluster.
\end{itemize}}


\cventry{2012--2013}{MRes (Distinction), Biodiversity Informatics and Genomics}{Imperial College London}{Silwood Park}{United Kingdom}{}

%\cvline{}{\textbf{Thesis:} Modelling and projecting response of Colombian biodiversity to land-use change}
%\cvline{}{Supervisor: Prof. Andy Purvis}

\cventry{2004--2010}{BSc (1st class honours) in Biology}{Universidad Industrial de Santander}{Bucaramanga}{Colombia}{}
%\cvline{}{Grade point average, 4.2/5.0}
%\cvline{}{\textbf{Thesis:} A geometrical approach for identifying geographical patterns of distribution.}
%\cvline{}{Supervisor: Dr Daniel Rafael Miranda-Esquivel}

\small
\section{COMPUTER SKILLS}
\cvline{OS}{Linux/Unix, Windows, OS X}{}{}
\cvline{Programming}{Proficient R, Basic Python, Basic C++, shell, basic HPC cluster scripts}{}{} 
\cvline{Data Science}{Data cleaning, processing and visualization, proficient GIS skills in R and ArcGIS, Advanced statistical analysis (e.g., Generalized Linear Mixed Models, Random Forests)}{}{}
\cvline{Version control}{Git}{}{}
\cvline{Typography}{\LaTeX, OpenOffice/LibreOffice}{}{}
%\cvcomputer{Phylogeny software}{POY, TNT, Nona, Winclada, PhyML, MrBayes, Bioedit, PAUP, R, BAMM}{}{}
%\cvcomputer{Biogeography software}{BioGeoBears, NDM, VNDM, Component 1.5, Component 2, DIVA, TREEFITTER, Martitracks}{}{}


\section{PUBLICATIONS}

\cvline{In prep}{\textbf{Echeverr\'ia-Londo\~no S.}, Neves, DM. , Kerkhoff, AJ, Enquist, BJ. Evolutionary biome shifts of angiosperms in the New World.}

\cvline{In prep}{Neves, DM. , Kerkhoff, AJ, \textbf{Echeverr\'ia-Londo\~no S.}, Enquist, BJ. The evolutionary history of New World biomes.}

\cvline{In prep}{\textbf{Echeverr\'ia-Londo\~no S.,} De Palma, A., Contu, S., Hill, S. L., Lysenko, I., Knapp, S., and Purvis, A. Plant species' responses to land-use change: A global perspective.}

\cvline{2018}{\textbf{Echeverr\'ia-Londo\~no S.,} Enquist, BJ, Neves, DM, Violle, C and Kerkhoff, AJ.  Plant functional diversity and the biogeography of biomes in North and South America, Frontiers in Ecology and Evolution, 6(DEC), 219.}

\cvline{2018}{\textbf{Echeverr\'ia-Londo\~no S,.}  S{\"a}rkinen, T., Fenton, I. S., Knapp, S., \& Purvis, A. Dynamism and context dependency in the diversification of the megadiverse plant genus Solanum L.(Solanaceae). bioRxiv, 348961.}

\cvline{2017}{Isaacs, P., \textbf{Echeverr\'ia-Londo\~no, S.,} Urbina, N and Purvis, A. Species Composition and Changes in Land Use: considerations under climatic change scenarios. Moreno, L. A., Andrade, G. I. and Ru\'iz-Contreras, L. F. (Ed.)  In \textit{Biodiversity 2016. Status and Trends of Colombian Continental Biodiversity}. Instituto de Investigaci\'on de Recursos Biol\'ogicos Alexander von Humboldt, 106 p.}

\cvline{2017}{Hudson, L. N., Newbold, T., Contu, S., Hill, S. L., Lysenko, I., De Palma, A., Diaz, S., \textbf{Echeverr\'ia-Londo\~no S,.} ... \& Purvis. The database of the Predicts (Projecting responses of ecological diversity in changing terrestrial systems) project. Ecology and Evolution, 7(1), 145-188.}

\cvline{2016}{\textbf{Echeverr\'ia-Londo\~no S.,} Newbold, T., Hudson, L. N., Hill, S. L., Contu, S., Lysenko, I., ... \& Purvis, A. Modelling and projecting the response of Colombian biodiversity to land-use change. Diversity and Distributions, 22: 1099-1111, doi:10.1111/ddi.12478.}

\cvline{2015}{Newbold, T., Hudson, L. N., Hill, S. L., Contu, S., Lysenko, I., Senior, R. A., Bennet D. J., Choimes A., Collen, B., Day, J., De Palma, A., Diaz, S., \textbf{Echeverr\'ia-Londo\~no S,.} ... \& Purvis, A. Global effects of land use on local terrestrial biodiversity. Nature, 520(7545), 45-50.}

\cvline{2014}{Newbold, T., Hudson, L. N., Contu, S.,  Hill, S. L., Lysenko, I., De Palma, A., Phillips, H., Senior, R. A., Bennet D. J., Booth, H., Choimes A., Correia, D. L. P., Day, J, \textbf{Echeverr\'ia-Londo\~no S,.} ... \& Purvis, A. The PREDICTS database: a global database of how local terrestrial biodiversity responds to human impacts. Ecology and evolution 4.24: 4701-4735. }

\cvline{2011}{\textbf{Echeverr\'ia-Londo\~no, S }\& Miranda-Esquivel, D. R. MartiTracks: A geometrical approach for identifying geographical patterns of distribution. PLoS ONE 6(4): e18460. doi:10.1371/journal.pone.0018460 \url{http://dx.plos.org/10.1371/journal.pone.0018460}.}

\cvline{2010}{Miranda-Esquivel, D. R, Morales-Guerrero, A. M \& \textbf{Echeverr\'ia-Londo\~no S.} \textquestiondown Cu\'antos somos? y \textquestiondown C\'omo nos cuantificamos?. Book chapter. Evoluci\'on "Historia de la Vida". Asociaci\'on Colombiana para el avance de la Ciencia ACAC.}


\section{INTERNATIONAL CONFERENCE PRESENTATIONS}

\cvline{2018}{\textbf{Presentation:} Echeverr\'ia-Londo\~no S, et al. Explosive diversification of \textit{Solanum} L (Solanaceae) in the old world. \textbf{Three Rivers Evolution meeting}, Pittsburgh, Pennsylvania, USA}

\cvline{2018}{\textbf{Poster:} Echeverr\'ia-Londo\~no S, et al. Plant functional diversity and the biogeography of biomes in North and South America. \textbf{Three Rivers Evolution meeting}, Pittsburgh, Pennsylvania, USA}

\cvline{2018}{\textbf{Poster:} Echeverr\'ia-Londo\~no S, et al. Plant functional diversity and the biogeography of biomes in North and South America. \textbf{Botany 2018}, Rochester, Minnesota, USA}

\cvline{2016}{\textbf{VVIP Presentation:} Echeverr\'ia-Londo\~no S, et al. Modelling and projecting the response of Colombian biodiversity to land-use change. \textbf{State visit by the President of Colombia, Juan Manuel Santos, to the UK and the Prince of Wales}. Natural History Museum, London, United Kingdom}

\cvline{2016}{\textbf{Presentation:} Echeverr\'ia-Londo\~no S, et al. Modelling and projecting the response of Colombian biodiversity to land-use change. \textbf{First symposium of Colombian research in the UK}. Imperial College London, London, United Kingdom}

\cvline{2016}{\textbf{Presentation:} Echeverr\'ia-Londo\~no S, Knapp S. \& Purvis A. Explosive diversification of \textit{Solanum} L (Solanaceae) in the old world. \textbf{BES macroecology meeting 2016}. Oxford University, Oxford, United Kingdom}

\cvline{2016}{\textbf{Presentation:} Echeverr\'ia-Londo\~no S, Knapp S. \& Purvis A. Diversification patterns of genus \textit{Solanum}. \textbf{NHM students conference 2016}. Natural History Museum, London, United Kingdom}

\cvline{2015}{\textbf{Poster:} Echeverr\'ia-Londo\~no S, Knapp S. \& Purvis A. Explosive diversification of \textit{Solanum} L (Solanaceae) in the old world. \textbf{Systematics: The Science that Underpins Biology. The Systematic Association Biennal meeting}, Oxford University, Oxford, United Kingdom}

\cvline{2015}{\textbf{Presentation:} Echeverr\'ia-Londo\~no S, Knapp S. \& Purvis A. Explosive diversification of \textit{Solanum} L (Solanaceae) in the old world. \textbf{EU Macroecology meeting}. Zoological museum of Copenhagen, Copenhagen, Denmark}

\cvline{2015}{\textbf{Poster:} Echeverr\'ia-Londo\~no S, Knapp S \& Purvis A. Diversification patterns of \textit{Solanum}. \textbf{NHM students conference 2015}. Natural History Museum, London, United Kingdom}

\cvline{2014}{\textbf{Poster:} Echeverr\'ia-Londo\~no S, Knapp S \& Purvis A. Diversification patterns of \textit{Solanum}. \textbf{The London Evolutionary Research Network (LERN)}. Queen Mary University, London, United Kingdom}

\cvline{2014}{\textbf{Poster:} Echeverr\'ia-Londo\~no S, Knapp S \& Purvis A. Patterns of lineage diversification in the genus Solanum L. \textbf{Plants Radiation meeting}. Institute of Systematic Botany, University of Z\"urich, Switzerland}

\cvline{2013}{\textbf{Presentation:} Echeverr\'ia-Londo\~no S, \textit{et al.,} Modelling and projecting the responses of Colombian biodiversity to human impacts. \textbf{11th INTECOL Conference, Ecology}, London, United Kingdom}

\cvline{2010}{\textbf{Presentation:} Echeverr\'ia-Londo\~no S \& Miranda-Esquivel D. M. Panbiogeograf\'ia cuantitativa: un acercamiento geom\'etrico. \textbf{III Congreso Colombiano de Zoolog\'ia.} Medell\'in, Colombia}

\cvline{2009}{\textbf{Presentation:} Miranda-Esquivel D. M, Echeverr\'ia-Londo\~no S. \& Morales-Guerrero A. M. Todas las secuencias son iguales, pero unas son mas iguales que otras. \textbf{III Simposio red colombiana de Biolog\'ia Evolutiva.-Colevol-} Cali, Colombia}


\section{INVITED SEMINARS}

\cvline{2018}{Ecology and Evolution seminars, Department of Biological Sciences, University of Pittsburgh, USA.}

\cvline{2017}{Department of Biology seminars, Kenyon College, USA.}

\cvline{2017}{Ecosystems and Tropical Forest group lab seminar, School of Geography and the Environment, University of Oxford, UK.}

\cvline{2015}{Ecology and Evolution seminar, Royal Botanic Garden Edinburgh, UK.}


\section{AWARDS AND RECOGNITIONS}

\cvline{2018}{NSF grant fund ``RCN-UBE Incubator: The Biological and Environmental Data Education Network'', \$ 72,361}

\cvline{2016}{Selected to present research to the (Nobel Peace Prize winning) President of Colombia and His Royal Highness the Prince of Wales. Natural History Museum, London, UK}

\cvline{2013}{Distinction in MRes in Biodiversity Informatics and Genomics. Imperial College London. Silwood Park Campus, UK}

\cvline{2004--2009}{Academic Stimuli. First place (Highest GPA) in I, II, IV, IX semester. Academic program of Biology. Universidad Industrial de Santander, Bucaramanga, Colombia}

\section{RESEARCH SCHOLARSHIPS}

\cvline{2012-2016}{Postgraduate scholarship. Administrative Department of Science, Technology and Innovation  of Colombia (Colciencias). $\pounds$122,000}
\cvline{2013}{Training and travel grant. British Ecology Society (BES): $\pounds$300}
\cvline{2013}{Bursary. MRes in Biodiversity Informatics and Genomics. Imperial College London. Silwood Park Campus: $\pounds$1,000}


%\section{TEACHING EXPERIENCE}

%\cventry{2012-2013}{PREDICTS team member}{Imperial College London}{Silwood park}{United Kingdom}{Projecting Responses of Ecological Diversity In Changing Terrestrial Systems (PREDICTS) project \url{http://www.predicts.org.uk/team.html}}


\section{SCIENCE OUTREACH}
\cvline{2014}{\textit{Science Uncovered at the Natural History Museum}. Presented the PREDICT project (\url{www.predicts.org.uk/}): Projecting Responses of Ecological Diversity In Changing Terrestrial Systems, and also the Solanaceae source project \url{www.solanaceaesource.org}}

\newpage

\section{NON-ACADEMIC SERVICES}

\cvline{2015--2016}{Secretary. Latin American Society, Imperial College London, London, UK. Responsible for running the fresher-fair, writing minutes and organizing dance sessions.}



%\section{Undergraduate Thesis}


%\section{Awards	}

%Academic Stimuli. First place (Highest GPA) in # semester. Academic Program of Biology. Universidad Industrial de Santander.


%\newpage



%\section{Postgraduate courses}
%
%\cvline{2013}{Postgraduate Course in C programming}
%\cvline{}{Imperial College London}
%
%\cvline{2009}{Introduction to evolutionary biogeography. Facultad de Ciencias Exactas y Naturales. Posgrado de Biolog\'ia. Universidad de Antioquia. Medell\'in, Colombia.}
%\cvline{supervisors}{Juan Jose Morrone Lupi, Tania Escalante Espinosa}


%\section{Other courses}
%
%\cvline{2011}{Introduction to web based programming (HTML and JAVASCRIPT). Servicio Nacional de Aprendizaje SENA}
%
%\cvline{2011}{Java applet development, events management, classes and objects. Servicio Nacional de Aprendizaje SENA.}
%
%\cvline{2011}{Modules, storage structures and OOP using C++ programming language( Level II). Servicio Nacional de Aprendizaje SENA.}
%
%\cvline{2010}{Structure oriented programming language C++ (Level I). Servicio Nacional de Aprendizaje SENA.}
%
%\cvline{2010}{Variables, and control structures in object oriented programming: JAVA. Servicio Nacional de Aprendizaje SENA.}



%\section{Training courses and workshops}
%
%\cvline{2014}{Plant taxonomy course. NERC-funded training courses. The Natural History Museum, London, United Kingdom}
%
%\cvline{2014}{Molecular techniques for taxonomy. NERC-funded training courses. The Natural History Museum, London, United Kingdom}
%
%\cvline{2014}{Radiation and extinction: Investigating clade dynamics in deep time workshop. The Linnean Society and Imperial College London, London, United Kingdom}
%
%\cvline{2014}{Phylogeny, extinction risks and conservation discussion meeting. The Royal Society, London, United Kingdom}
%
%\cvline{2013}{Spatial analysis in R. BES Macroecology SIG workshop. Sheffield University, Sheffield, United Kingdom}
%
%\cvline{2013}{Postgraduate course in C programming, Imperial College London, London, United Kingdom}
%
%\cvline{2013}{The generalised linear modelling (GLIM) course presented by Prof. M. J. Crawley, Imperial College London, London, United Kingdom}
%
%\cvline{2009}{Introduction to evolutionary biogeography. Facultad de Ciencias Exactas y Naturales. Posgrado de Biolog\'ia. Universidad de Antioquia. Medell\'in, Colombia.}



\section{LANGUAGES}
\cvline{Spanish}{Native}
\cvline{English}{Proficient}
\cvline{Italian}{Intermediate}



%\cvline{2015}{International development conference (IDC). Imperial College London, London, United Kingdom}
%
%\cvline{2014}{16th Young Systematists' forum. Natural History Museum, London, United Kingdom}
%
%
%
%\cvline{2014}{Plants Radiation meeting. Institute of Systematic Botany, University of Z\"urich, Switzerland}
%
%\cvline{2013}{11th INTECOL Congress, Ecology, London, United Kingdom}
%
%\cvline{2010}{III Congreso Colombiano de Zoolog\'ia. Medell\'in, Colombia.}
%\cvline{2009}{III Simposio red colombiana de Biolog\'ia Evolutiva.-Colevol-. Cali-Colombia}
%\cvline{2007}{V Congreso Internacional y VIII Congreso Colombiano de Gen\'etica. Cali, Colombia.}
%\cvline{2007}{I Congreso Nacional de Estudiantes de Biolog\'ia. Bogot\'a. Colombia.}
%\cvline{2006}{IV Congreso Internacional y VII Congreso Colombiano de Gen\'etica.  Bucaramanga, Colombia.}
%
%\section{\textbf{Languages}}
%\cvlanguage{2012}{Pre-sessional course in English.}{Level of competence in English: 7.0 \\ University of Bath}

%\cvlanguage{2011}{TOEFL IBT}{
%Score 98/120\\
%Reading score: 29\\
%Listening score: 25\\
%Speaking score: 23\\
%Writing score: 21\\
%ETS ID:	5653379\\
%Registration Number: 0000000013635906\\
%Date of examination: 17-Dec-2011
%}
%\cvlanguage{2010}{Michigan English Test}{Score 127}

\section{References}

\vspace{0.3cm}

\cvreference{Prof Andrew J. Kerkhoff}
{Professor of Biology}
{Department of Biology}
{Kenyon College}
{Higley Hall, Gambier, OH 43022}
{+1 740 427 5734}
{kerkhoffa@kenyon.edu}%

\vspace{0.5cm}
    
\cvreference{Prof Andy Purvis}
 	{Research Leader}
    {Department of Life Science}
    {Natural History Museum and Imperial College London}
    {Cromwell Rd, Kensington, London SW7 5BD}
    {Andy.Purvis@nhm.ac.uk}
    {+44 020 7942 5686}%
    
    \vspace{0.5cm}
    
\cvreference{Dr Sandra Knapp}
    {Merit Researcher, Head of Division}
    {Department of Life Science}
    {Natural History Museum}
    {Cromwell Rd, Kensington, London SW7 5BD}
    {+44 020 7942 5171}
    {s.knapp@nhm.ac.uk}%
    
      \vspace{0.5cm}
    


\closesection{}                   % needed to renewcommands
\renewcommand{\listitemsymbol}{-} % change the symbol for lists

%\section{Extra 1}
%\cvlistitem{Item 1}
%\cvlistitem{Item 2}
%\cvlistitem[+]{Item 3}            % optional other symbol

%\section{Extra 2}
%\cvlistdoubleitem[\Neutral]{Item 1}{Item 4}
%\cvlistdoubleitem[\Neutral]{Item 2}{Item 5}
%cvlistdoubleitem[\Neutral]{Item 3}{}

% Publications from a BibTeX file
%\nocite{*}
%\bibliographystyle{plain}
%\bibliography{publications}       % 'publications' is the name of a BibTeX file

\end{document}


%% end of file `template_en.tex'.