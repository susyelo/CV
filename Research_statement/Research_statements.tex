% Upper-case    A B C D E F G H I J K L M N O P Q R S T U V W X Y Z
% Lower-case    a b c d e f g h i j k l m n o p q r s t u v w x y z
% Digits        0 1 2 3 4 5 6 7 8 9
% Exclamation   !           Double quote "          Hash (number) #
% Dollar        $           Percent      %          Ampersand     &
% Acute accent  '           Left paren   (          Right paren   )
% Asterisk      *           Plus         +          Comma         ,
% Minus         -           Point        .          Solidus       /
% Colon         :           Semicolon    ;          Less than     <
% Equals        =           Greater than >          Question mark ?
% At            @           Left bracket [          Backslash     \
% Right bracket ]           Circumflex   ^          Underscore    _
% Grave accent  `           Left brace   {          Vertical bar  |
% Right brace   }           Tilde        ~

%
% NSF Biographical Sketch: Dr. Theodore P. Pavlic
%
% Copyright (c) 2013 by Theodore P. Pavlic
% All rights reserved.
%

  \documentclass[svgnames,11pt,longbibliography,reqno]{article}
  \usepackage[margin=1in]{geometry}
  \usepackage[T1]{fontenc}
  \usepackage{times}
  \renewcommand\thesection{(\alph{section})}
  \renewcommand\thesubsection{(\alph{subsection})}

  \usepackage{calc}

  \usepackage{lastpage}
  %\newcommand{\pageoflastpage}{Page {\thepage} of \pageref*{LastPage}}
  %\newcommand{\pageoflastpage}{{\thepage} of \pageref*{LastPage}}
  \newcommand{\pageoflastpage}{}

  % Redefine the plain pagestyle for the title page
  \makeatletter
  \let\oldps@plain\ps@plain
  \renewcommand{\ps@plain}{\oldps@plain%
  \renewcommand{\@evenfoot}{\hfil\pageoflastpage\hfil}%
  \renewcommand{\@oddfoot}{\@evenfoot}}
  \makeatother

  % Use fancy for non-title pages
  \usepackage{fancyhdr}
  \fancyhead{}
  \fancyfoot{}
  \cfoot{\pageoflastpage}
  \pagestyle{fancy}

  \usepackage{xspace}

  \usepackage[shortlabels]{enumitem}
  \usepackage{graphicx}
  \usepackage{varioref}
  \usepackage[%
          colorlinks=true,urlcolor=,
          pdfpagelabels=true,hypertexnames=true,
          plainpages=false,naturalnames=true,
          ]{hyperref}
  %
  \usepackage{url}
  \newcommand\doilink[1]{\href{http://dx.doi.org/#1}{#1}}
  \newcommand\doi[1]{doi:\doilink{#1}}

  \makeatletter
  \newlength{\bibhang}
  \setlength{\bibhang}{1em}
  \newlength{\bibsep}
   {\@listi \global\bibsep\itemsep \global\advance\bibsep by\parsep}
  \newlist{bibsection}{itemize}{3}
  \setlist[bibsection]{label=,leftmargin={\bibhang+\widthof{[9]}},%
          itemindent=-\bibhang,
          itemsep=\bibsep,parsep=\z@,partopsep=0pt,
          topsep=0pt}
  \newlist{bibenum}{enumerate}{3}
  \setlist[bibenum]{label=\textbf{\arabic*.},leftmargin={\bibhang+\widthof{[999]}},%
          itemindent=-\bibhang,
          itemsep=\bibsep,parsep=\z@,partopsep=0pt,
          topsep=0pt}
  \let\oldendbibenum\endbibenum
  \def\endbibenum{\oldendbibenum\vspace{-.6\baselineskip}}
  \let\oldendbibsection\endbibsection
  \def\endbibsection{\oldendbibsection\vspace{-.6\baselineskip}}
  \makeatother

  \title{%
          \vspace{-2\baselineskip}
              \normalsize
              \large Research statement\\
              {\Large \textbf{Susy Echeverr\'ia-Londo\~no}}\\
              \vspace{0.5\baselineskip}
              \hrule
              \vspace{0.5\baselineskip}
              Department of Life Science, Imperial College London
              e-mail: \href{mailto:se1212@ic.ac.uk}{se1212@ic.ac.uk},
              \href{mailto:susye@nhm.ac.uk}{susye@nhm.ac.uk},
              tel: +44-079-2578-6803
          \vspace{-1.5ex}
          }
  \date{}
  \author{}

  \hypersetup{%
          pdfsubject={Postdoctoral applications},
          pdfauthor={Susy Echeverr\'ia-Londo\~no},
          pdftitle={Research statement},
        pdfkeywords={}
        }

\newcommand{\ie}{i.e.\xspace}
\newcommand{\eg}{e.g.\xspace}

\begin{document}

\maketitle
\vspace{-2\baselineskip}

Our knowledge of the current state of biodiversity, usually derived by local-scale sampling, is being outpaced by the widespread changes driven by anthropogenic pressures. For this reason, it is essential to develop integrated approaches to consistently and rapidly monitoring large-scale patterns of diversity, especially in species-rich tropical regions where traditional practices of monitoring could fail to overcome the current changes in diversity. Given how rapidly biodiversity data are now accumulating, my research goal is to integrate diverse biodiversity data (e.g., phylogenetic relationships, species distributions and functional traits) to understand the dynamics that have shaped the current patterns of diversity and eventually predict how the biota will respond to accelerating environmental change. In this context, my research work can be divided into two different projects: 

\subsection*{Biodiversity responses to land-use change}

Land-use change is the current most significant threat to terrestrial biodiversity. Global models of local-scale responses to habitat conversion can help us to quantify and project future changes of biodiversity at national and global scales. My research interest in this area has arisen mainly from my experience working on the PREDICTS project,  \href{http://www.predicts.org.uk/}{www.predicts.org.uk}, which aims to investigate how local biodiversity typically responds to land-use change. In this project, my work focused on understanding the responses of Colombian biodiversity to land-use change using a collation of field data from different taxonomic groups and regions. During my involvement in the project, I contributed to the assembly of an extensive database of local biodiversity responses to land-use change (see Hudson et al., 2014 and Hudson et al., 2016), to model the global effects of land use on local terrestrial biodiversity (see Newbold et al., 2014) and to quantify past and possible future local biodiversity losses under land-use change in Colombia (see Echeverr\'ia-Londo\~no et al., 2016). This work has been of interest to the public and policymakers alike. In particular, I had the privilege to present the findings of my paper on Colombian biodiversity change to the (Nobel Peace Prize-winning) President of Colombia and His Royal Highness the Prince of Wales at the Natural History Museum in London, during the President's state visit to the UK. Over the next few years, I am to extend this work to improve the robustness of models of how habitat conversion affects Colombian biodiversity by more accurately accounting for differences among regions and taxonomic groups, which could help to inform mitigation policies of land-use change.  



\subsection*{The genus \textit{Solanum} as a case study for evolutionary dynamics of angiosperms}

\textit{Solanum} L. (Solanaceae) is one of the largest genera (ca. 1200 species) of angiosperms and is economically very important. Solanum's variety of traits, widespread distribution and variety of niches makes it an interesting case study for understanding clade diversification dynamics. My research in this area is focused on understanding the large-scale patterns of the diversity of \textit{Solanum} and the evolutionary and ecological processes that generate them. This work is split into three main sections: (1) using tools for modeling complex evolutionary dynamics, I am investigating how heterogeneity in diversification rates across phylogeny and time reflect biogeographic events and historical environmental changes, and shape patterns of diversity across the extant lineages of \textit{Solanum} (Echeverr\'ia-Londo\~no et al, 2018). (2) By combining morphological, ecological, taxonomic and phylogenetic information, I also attempt to understand the ecological context of radiation of \textit{Solanum}. (3) I am also currently analysing whether macroevolutionary inferences are robust to the fact that the clade's taxonomy has reached different levels of maturity in different regions and lineages. I intend to extend this research to other groups of flowering plants. Understanding the evolutionary dynamics of recent radiations such as \textit{Solanum} provides an excellent opportunity to understand the origin and phylogenetic and biogeographic distribution of biodiversity. For this reason, I hope to continue filling critical knowledge gaps in the diversification of this genus to build more robust generalisation eventually on the evolutionary processes that promote biological diversity.


\subsection*{Future research plans}

So far, I have pursued these two lines of research independently. However, in my next research project, I would like to combine the tools and ideas from both research fields, for example, using macroevolutionary tools to understand the responses of communities to changing environments.

Biogeographic origins and historical environmental changes have shaped the diversity and composition of regional biotas and the ecosystem services they provide. In general, we expect that lineages shaped by contrasting historical events to respond differently to resource limitation and changing environmental factors. It is for this reason that I believe the study of historic events such as biogeographic history, past disturbance regimes, and climatic instability can provide a predictive framework to guide ecosystem management. In particular, I am interested in linking biogeographic and evolutionary history with the functional composition and diversity of plant communities, which ultimately are the foundation for understanding the productivity of ecosystems. Assuming that the persistence of ancestral traits from former evolutionary regimes influences the microhabitats that lineages currently occupy, we can implement conservation practices to increase the diversity of species with different evolutionary histories. In this way, we can contribute the complementarity in resource use and eventually increase the resilience and productivity of ecosystems. However, this is a relatively unexplored field and more research is needed to understand in more detail the dependence of current ecosystem functions on historical legacies. Therefore, my initial goal in this area is to identify signatures of the imprint of historical factors by comparing the current lineage composition, diversity and functional trait distributions in regions with similar climatic conditions and geological settings. To do this, I will integrate approaches from macroevolution, macroecology and phylogenetic community theory. 


\vspace{\baselineskip}

\small 

\begin{bibenum}[itemsep=5pt]


\item \textbf{Echeverr\'ia-Londo\~no S}., Newbold, T., Hudson, L. N., ... \& Purvis, A. (2016) Modelling and projecting the response of Colombian biodiversity to land-use change. \textit{Diversity and Distributions}, 22: 1099-1111. 

\item \textbf{Echeverr\'ia-Londo\~no} S., et al.  (2018) Dynamism and context-dependency in the diversification of megadiverse plant groups. bioRxiv: 348961.

 \item Hudson, L. N., Newbold, T., Contu, S., Hill, S. L., Lysenko, I., De Palma, A., Phillips, H., Senior, R. A., Bennet D. J., Booth, H., Choimes A., Correia, D. L. P., Day, J, \textbf{Echeverr\'ia-Londo\~no} S,. ... \& Purvis, A (2016) The PREDICTS database: a global database of how local terrestrial biodiversity responds to human impacts. \textit{Ecology and Evolution}, http://onlinelibrary.wiley.com/doi/10.1002/ece3.2579/full

 \item Hudson, L. N., Newbold, T., Contu, S., Hill, S. L., Lysenko, I., De Palma, A., Phillips, H., Senior, R. A., Bennett D. J., Booth, H., Choimes A., Correia, D. L. P., Day, J, \textbf{Echeverr\'ia-Londo\~no} S,. ... \& Purvis, A (2015) The PREDICTS database: a global database of how local terrestrial biodiversity responds to human impacts. \textit{Ecology and Evolution}, 4.24: 4701-4735.

 \item Newbold, T., Hudson, L. N., Hill, S. L., Contu, S., Lysenko, I., Senior, R. A., Bennett D. J., Choimes A., Collen, B., Day, J., De Palma, A., Diaz, S., \textbf{Echeverr\'ia-Londo\~no} S,. ... \& Purvis, A. (2014) Global effects of land use on local terrestrial biodiversity. \textit{Nature}, 520(7545), 45-50.



\end{bibenum}

\end{document}