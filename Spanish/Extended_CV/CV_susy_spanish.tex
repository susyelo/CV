\documentclass[11pt,a4paper]{moderncv}
\usepackage[latin1]{inputenc}
\usepackage{url}


% moderncv themes
%\moderncvtheme[blue]{casual}    		% optional argument are 'blue' (default), 'orange', 'red', 'green', 'grey' and 'roman' (for roman fonts, instead of sans serif fonts)
\moderncvtheme[green]{classic}                	% idem

% character encoding
%\usepackage[utf8]{inputenc}                   	% replace by the encoding you are using

% adjust the page margins
\usepackage[right=1.7cm,left=1.7cm,textheight=24.5cm]{geometry}
\recomputelengths                             		% required when changes are made to page layout lengths
\setlength{\hintscolumnwidth}{2.5cm}

% personal data
\firstname{\huge Susy}
\familyname{\huge Echeverr\'ia-Londo\~no, PhD }
\title{\huge Investigadora Asociada \vspace{0.1cm} \newline \large  Vaccine Impact Modelling Consortium (VIMC)  \newline Imperial College London}	% optional, remove the line if not wanted
\address{St Mary's hospital}{Londres, Reino Unido \vspace{0.1cm}}{}    	% optional, remove the line if not wanted
% \mobile{+44 7925786803}                    							% optional, remove the line if not wanted
\email{susy.echeverria-londono@imperial.ac.uk}									% optional, remove the line if not wanted
\social[github]{https://github.com/susyelo}  
\extrainfo{\href{https://susyelo.github.io/}{https://susyelo.github.io/}}                    			% optional, remove the line if not wanted

%\nopagenumbers{}                             % uncomment to suppress automatic page numbering for CVs longer than one page

\newcommand\myitem{\item[\textbullet]}

\newcommand{\cvreference}[7]{%
    \textbf{#1}\newline% Name
    \ifthenelse{\equal{#2}{}}{}{\addresssymbol~#2\newline}%
    \ifthenelse{\equal{#3}{}}{}{#3\newline}%
    \ifthenelse{\equal{#4}{}}{}{#4\newline}%
    \ifthenelse{\equal{#5}{}}{}{#5\newline}%
    \ifthenelse{\equal{#6}{}}{}{\emailsymbol~\texttt{#6}\newline}%
    \ifthenelse{\equal{#7}{}}{}{\phonesymbol~#7}}
%----------------------------------------------------------------------------------
%            content
%----------------------------------------------------------------------------------
\begin{document}
\maketitle


%\vspace{-0.25cm}
%\section{SUMMARY}
%
%I am a researcher focusing on the study of large-scale patterns of biodiversity, and the ecological and evolutionary processes that shape them. Throughout my career, I have gained extensive experience in data science, including data processing, data visualisation and advanced statistical and spatial analysis. I particularly enjoy exploring and analysing biological data through visualisation tools.


\vspace{0.25cm}
\section{EXPERIENCIA PROFESIONAL}

\cventry{2019--Presente}{Investigadora asociada}{Consorcio de modelamiento del impacto de vacunas. \href{https://www.vaccineimpact.org/secretariat/}{https://www.vaccineimpact.org/secretariat/} Imperial College London}{}{Reino Unido}{}
\cvline{}{PI: Prof. Neil Ferguson}

\cventry{2018--2019}{Investigadora visitante}{Universidad de Pittsburgh, Ciencias biol\'ogicas}{}{Estados Unidos}{}
\cvline{}{PI: Prof. Justin Kitzes.}
%\cvline{}{Exploring and analysing time-series spatial points patterns from ca. 300 plant species to predict extinction risks.}

\cventry{2017-2019}{Investigadora postdoctoral asociada}{Kenyon College, Biology}{}{Gambier, Estados Unidos}{}
\cvline{}{PIs: Prof. Andrew Kerkhoff (Kenyon College) and Prof. Brian J. Enquist (Universidad de Arizona)}
\cvline{}{Proyecto: Evoluci\'on de nichos, l\'imites ecol\'ogicos y la macroecolog\'ia de plantas en el Neotr\'opico y el Ne\'artico. Financiado por la fundaci\'on nacional de ciencias de Estados Unidos (NSF)}
%\cvline{}{%
%	\begin{itemize}
%		\item Exploring and analysing the distribution of ca. 85000 plant species, including approximately 9 million geographic points.
%		\item Researching the consequences of habitat stability on current patterns of plant diversity using distribution and phylogenetic relations of ca. 24000 plant species.
%		\item Carrying out paleohabitat reconstructions using machine learning methods.
%		\item Co-taught the BSc course ``Global Ecology and Biogeography''.
%		\item Taught and organised lectures and assignments for the undergraduate ecoinformatics course, which include topics such as ``Introduction to data science R'', ``Data managing and processing'', ``Data visualisation'', ``Spatial analysis and reproducibility'' (see \href{https://globalecologybiogeography.github.io/Ecoinformatics/}{https://globalecologybiogeography.github.io/Ecoinformatics/}). \vspace{-0.5cm}
%		\end{itemize}}
\cventry{2017}{Instructora y docente}{Natural History Museum \& Imperial College London}{}{Reino Unido}{}
\cvline{}{Instructora y docente del curso ``M\'etodos en macroecolog\'ia y macroevoluci\'on'' para la maestria MSc ``Taxonomy and Biodiversity''}

\cventry{2013--2016}{Instructora}{Imperial College London}{}{Reino Unido}{}
\cvline{}{Instructora de cursos de pregrado como ``Bioestad\'istica computacional'', ``Ecolog\'ia y evoluci\'on'', ``Biodiversidad y biolog\'ia de la conservaci\'on'', ``Introducci\'on a R'', ``Fundamentos en estad\'istica en R''}
%
\cventry{2011--2012}{Docente de Ciencias}{New Cambridge school, Gimnasio pontevedra, La quinta del puente}{Bucaramanga}{Colombia}{}
\cvline{}{Docente de Ciencias y biolog\'ia para estudiantes de bachillerato y primaria}
%
\cventry{2008--2010}{Docente auxiliar}{Universidad Industrial de Santander}{}{Bucaramanga, Colombia}{}
\cvline{}{Docente auxiliar en los cursos de pregrado en Sistem\'atica y evoluci\'on y el curso de Maestr\'ia en Bioinformatica}


%\vspace{0.25cm}

\section{EDUCACI\'ON}

\cventry{2013--2017}{Doctorado en Ciencias de la Vida}{Imperial College London, Natural History Museum}{}{Londres, Reino Unido}{}

\cvline{}{Proyecto: Patrones de diversificaci\'on del g\'enero \textit{Solanum} L. (Solanaceae), macroecolog\'ia de plantas y efectos de los cambios del uso de habitat.}
\cvline{}{Supervisor: Prof. Andy Purvis. Co-supervisor Dr Sandra Knapp.}
%\cvline{}{%
%	\begin{itemize}
%		\item Researched the dynamics of diversification of the megadiverse plant genus \textit{Solanum} using a collation of geographical, genetic and taxonomic data along with phylogenetic comparative methods.
%		\item Studied large-scale patterns of plant diversity and their responses to land-use change from data of a global collation of local field-based studies.
%		\item Carried out large-scale evolutionary modelling exercise on an HPC Linux cluster. \vspace{-0.5cm}
%\end{itemize}}


\cventry{2012--2013}{Master de Investigaci\'on en Biodiversidad, Inform\'atica y Gen\'omica \normalfont{(Estudiante distinguida)}}{Imperial College London}{}{Silwood Park, Ascot, Reino Unido}{}
\cvline{}{Proyecto: Modelamiento y proyecci\'on de respuestas de la biodiversidad Colombiana a cambios del uso de habitat}
\cvline{}{Supervisor: Prof. Andy Purvis.}

\cventry{2004--2010}{Pregrado en Biolog\'ia}{Universidad Industrial de Santander}{}{Bucaramanga, Colombia}{}
%\cvline{}{Grade point average, 4.2/5.0}
%\cvline{}{\textbf{Thesis:} A geometrical approach for identifying geographical patterns of distribution.}
%\cvline{}{Supervisor: Dr Daniel Rafael Miranda-Esquivel}

%\clearpage
\section{PUBLICACIONES}

\cvline{En revisi\'on}{Xiang Li, Christinah Mukandavire, \textbf{Susy Echeverr\'ia-Londo\~no}, Zulma M Cucunub\'a, Kaja Abbas, Hannah E Clapham, ... , Neil M Ferguson, Tini Garske, (Vaccine Impact Modelling Consortium).  Estimating the health impact of vaccination against 10 pathogens in 98 low and middle income countries from 2000 to 2030. Sometido a la revista Lancet.}

\cvline{2020}{Laiton-Donato K, Villabona Arenas CJ, Usme Ciro JA, Franco Munoz C, Alvarez-Diaz DA, Villabona-Arenas LS, \textbf{Echeverria-Londono S}, Franco-Sierra ND, Cucunuba ZM, Florez-Sanchez AC, Ferro C, Ajami NJ, Walteros DM, Prieto-Alvarado FE, Duran-Camacho CA, Ospina-Martinez ML, Mercado-Reyes MM. Genomic epidemiology of SARS-CoV-2 in Colombia. Aceptado en la revista \textit{Emerging Infectious Diseases}. doi:10.1101/2020.06.26.20135715. PPR:PPR181191.}

\cvline{2020}{\textbf{Echeverr\'ia-Londo\~no, S,.}  S{\"a}rkinen, T., Fenton, I. S., Knapp, S. and Purvis, A. Dynamism and context dependency in the diversification of the megadiverse plant genus Solanum L. (Solanaceae), \textit{Journal of Systematics and Evolution}. https://doi.org/10.1111/jse.12638}

\cvline{2018}{\textbf{Echeverr\'ia-Londo\~no, S.,} Enquist. B. J., Neves, D. M., Violle, C. and Kerkhoff, A. J. Plant functional diversity and the biogeography of biomes in North and South America, \textit{Frontiers in Ecology and Evolution}, 6(DEC), 219.}

\cvline{2017}{Isaacs, P., \textbf{Echeverr\'ia-Londo\~no, S.,} Urbina, N. and Purvis, A. Species composition and changes in land use: considerations under climatic change scenarios. Moreno, L. A., Andrade, G. I. and Ru\'iz-Contreras, L. F. (Editors). In \textit{Biodiversity 2016. Status and Trends of Colombian Continental Biodiversity}, Instituto de Investigaci\'on de Recursos Biol\'ogicos Alexander von Humboldt, 106 p.}

\cvline{2017}{Hudson, L. N., Newbold, T., Contu, S., Hill, S. L., Lysenko, I., De Palma, A., Diaz, S., \textbf{Echeverr\'ia-Londo\~no, S.,} \dots and Purvis, A. The database of the PREDICTS (Projecting Responses of Ecological Diversity in Changing Terrestrial Systems) project, \textit{Ecology and Evolution}, 7(1), 145--188.}

\cvline{2016}{\textbf{Echeverr\'ia-Londo\~no, S.,} Newbold, T., Hudson, L. N., Hill, S. L., Contu, S., Lysenko, I., \dots and Purvis, A. Modelling and projecting the response of Colombian biodiversity to land-use change, \textit{Diversity and Distributions}, 22, 1099--1111.}

\cvline{2015}{Newbold, T., Hudson, L. N., Hill, S. L., Contu, S., Lysenko, I., Senior, R. A., Bennet D. J., Choimes A., Collen, B., Day, J., De Palma, A., Diaz, S., \textbf{Echeverr\'ia-Londo\~no, S.,} \dots and Purvis, A. Global effects of land use on local terrestrial biodiversity, \textit{Nature}, 520(7545), 45--50.}

\cvline{2014}{Newbold, T., Hudson, L. N., Contu, S.,  Hill, S. L., Lysenko, I., De Palma, A., Phillips, H., Senior, R. A., Bennet D. J., Booth, H., Choimes, A., Correia, D. L. P., Day, J., \textbf{Echeverr\'ia-Londo\~no, S.,} \dots and Purvis, A. The PREDICTS database: a global database of how local terrestrial biodiversity responds to human impacts, \textit{Ecology and Evolution}, 4(24), 4701--4735. }

\cvline{2011}{\textbf{Echeverr\'ia-Londo\~no, S.} and Miranda-Esquivel, D. R. MartiTracks: A geometrical approach for identifying geographical patterns of distribution, \textit{PLoS ONE}, 6(4), e18460.}

\cvline{2010}{Miranda-Esquivel, D. R., Morales-Guerrero, A. M. and \textbf{Echeverr\'ia-Londo\~no, S.} \textquestiondown Cu\'antos somos? y \textquestiondown C\'omo nos cuantificamos? In \textit{Evoluci\'on ``Historia de la Vida". Asociaci\'on Colombiana para el avance de la Ciencia ACAC.}}


\vspace{0.25cm}
\section{SEMINARIOS Y CURSOS DICTADOS}

\cvline{2019}{Ponencia titulada \textbf{Experiencias de formaci\'on fuera de Colombia}. Presentado en el marco del evento ``Egresada destacada de la Escuela de Biolog\'ia''. Universidad Industrial de Santander, Colombia}

\cvline{2019}{Ponencia titulada \textbf{Macroecolog\'ia, macroevoluci\'on y cambios de la biodiversidad por usos de la tierra}. Presentado en el marco del evento ``Egresada destacada de la Escuela de Biolog\'ia''. Universidad Industrial de Santander, Colombia}

\cvline{2019}{Curso de \textbf{manipulaci\'on de datos espaciales en R}. Dise\~nado e impartido para 26 estudiantes, incluyendo estudiantes de pregrado, maestri\'ia y docentes universitarios. Universidad Industrial de Santander, Colombia}

\cvline{2018}{Ponencia titulada \textbf{``Plant functional diversity and the biogeography of biomes in North and South America''}. Presentado en el seminario de Ecolog\'ia y Evoluci\'on del Departamento de Ciencias Biol\'ogicas de la Universidad de Pittsburgh, PA, USA.}

\cvline{2017}{Ponencia titulada \textbf{``Explosive diversification of \textit{Solanum} L (Solanaceae) in the old world''}. Presentado en el seminario mensual del Departamento de Biolog\'ia en Kenyon College, Gambier, OH, USA.}

\cvline{2017}{Ponencia titulada \textbf{``Explosive diversification of \textit{Solanum} L (Solanaceae) in the old world''}. Presentado en el seminario del grupo de investigaci\'on de Ecosistemas y bosques tropicales de la Escuela de Geograf\'ia y Medioambiente de la Universidad de Oxford, Reino Unido.}

\cvline{2015}{Ponencia titulada \textbf{``Modelling and projecting the response of Colombian biodiversity to land-use change''}. Presentado en el simposio PREDICTS del Museo de historia natural de Londres, evento especial para celebrar los primeros tres a\~nos del proyecto PREDICTS. Ponencia disponible en \href{https://www.predicts.org.uk/predicts-symposium.html}{https://www.predicts.org.uk/predicts-symposium.html}}

\cvline{2015}{Ponencia titulada \textbf{``Patterns of lineage diversification in the genus \textit{Solanum} L.''}. Presentado en el seminario de Ecolog\'ia y Evoluci\'on del Jard\'in Bot\'anico Real de Edimburgo, Edimburgo, Reino Unido.}



\vspace{0.25cm}
\section{EVENTOS CIENT\'IFICOS}

\cvline{2018}{\textbf{Ponencia}. Echeverr\'ia-Londo\~no, S., et al. ``Explosive diversification of \textit{Solanum} L (Solanaceae) in the old world'', \textit{Three Rivers Evolution meeting}, Pittsburgh, USA.}

\cvline{2018}{\textbf{Poster}. Echeverr\'ia-Londo\~no, S., et al. ``Plant functional diversity and the biogeography of biomes in North and South America'', \textit{Three Rivers Evolution meeting}, Pittsburgh, USA.}

\cvline{2018}{\textbf{Poster}. Echeverr\'ia-Londo\~no, S., et al. ``Plant functional diversity and the biogeography of biomes in North and South America'', \textit{Botany 2018}, Rochester (MN), USA}

\cvline{2016}{\textbf{Ponencia VVIP}. Echeverr\'ia-Londo\~no, S., et al. ``Modelling and projecting the response of Colombian biodiversity to land-use change'', \textit{State visit by the President of Colombia, Juan Manuel Santos, to the UK and the Prince of Wales}, Natural History Museum, Londres, Reino Unido}

\cvline{2016}{\textbf{Ponencia}. Echeverr\'ia-Londo\~no, S., et al. ``Modelling and projecting the response of Colombian biodiversity to land-use change'', \textit{First symposium of Colombian research in the UK}, Imperial College London, Reino Unido.}

\cvline{2016}{\textbf{Ponencia}. Echeverr\'ia-Londo\~no, S., Knapp, S. and Purvis, A. ``Explosive diversification of \textit{Solanum} L (Solanaceae) in the old world'', \textit{BES macroecology meeting 2016}, Universidad de Oxford, Reino Unido. }

\cvline{2016}{\textbf{Ponencia}. Echeverr\'ia-Londo\~no, S., Knapp, S. and Purvis, A. ``Diversification patterns of genus \textit{Solanum}'', \textit{NHM students conference}, Natural History Museum, Londres, Reino Unido.}

\cvline{2015}{\textbf{Poster}. Echeverr\'ia-Londo\~no, S., Knapp, S. and Purvis, A. ``Explosive diversification of \textit{Solanum} L (Solanaceae) in the old world'', \textit{Systematics: The Science that Underpins Biology. The Systematic Association Biennal meeting}, Universidad de Oxford, Reino Unido.}

\cvline{2015}{\textbf{Ponencia}. Echeverr\'ia-Londo\~no, S., Knapp, S. and Purvis, A. ``Explosive diversification of \textit{Solanum} L (Solanaceae) in the old world'', \textit{EU Macroecology meeting}, Museo zool\'ogico de Copenhagen, Dinamarca.}

\cvline{2015}{\textbf{Poster}. Echeverr\'ia-Londo\~no, S., Knapp, S. and Purvis, A. ``Diversification patterns of \textit{Solanum}'', \textit{NHM students conference 2015}, Natural History Museum, Londres, Reino Unido.}

\cvline{2014}{\textbf{Poster}. Echeverr\'ia-Londo\~no, S., Knapp, S. and Purvis, A. ``Diversification patterns of \textit{Solanum}'', \textit{The London Evolutionary Research Network (LERN)}, Universidad de Queen Mary, Londres, Reino Unido.}

\cvline{2014}{\textbf{Poster}. Echeverr\'ia-Londo\~no, S., Knapp, S. and Purvis, A. ``Patterns of lineage diversification in the genus \textit{Solanum} L.'', \textit{Plants Radiation meeting}, Institute of Systematic Botany, Universidad de Z\"urich, Suiza.}

\cvline{2013}{\textbf{Ponencia}. Echeverr\'ia-Londo\~no, S., et al. ``Modelling and projecting the responses of Colombian biodiversity to human impacts'', \textit{11th INTECOL conference}, Londres, Reino Unido.}

\cvline{2010}{\textbf{Ponencia}. Echeverr\'ia-Londo\~no, S. and Miranda-Esquivel, D. M. Panbiogeograf\'ia cuantitativa: un acercamiento geom\'etrico, \textit{III Congreso Colombiano de Zoolog\'ia}, Medell\'in, Colombia.}


\vspace{0.25cm}
\section{RECONOCIMIENTOS}

\cvline{2018}{Beca de investigaci\'on, \$72,361, ``RCN-UBE Incubator: The Biological and Environmental Data Education Network'', \textit{NSF}.}

\cvline{2016}{Seleccionada para representar la investigaci\'on del Museo de Historia Natural ante la visita del Presidente de Colombia y ganador del premio Nobel de Paz Juan Manuel Santos y el pr\'incipe de Gales, \textit{Natural History Museum}, Londres, Reino Unido.}

\cvline{2013}{Subsidio de viaje, $\pounds$300, \textit{British Ecology Society}.}

\cvline{2013}{Beca acad\'emica, $\pounds$1,000, MRes in Biodiversity Informatics and Genomics, \textit{Imperial College London}, Silwood Park, Reino Unido.}

\cvline{2012--2016}{Credito-Beca para estudios doctorales, $\pounds$122,000, \textit{Departamento Administrativo de Ciencia, Tecnolog\'ia e Innovaci\'on (Colciencias)}.}

\cvline{2004--2009}{Incentivos acad\'emicos, Escuela de Biolog\'ia (Mejor promedio en los semestres I, II, IV, IX), \textit{Universidad Industrial de Santander}, Bucaramanga, Colombia.}


\vspace{0.25cm}
\clearpage

\section{OTRAS RESPONSABILIDADES}

\cvline{2014}{Divulgaci\'on cient\'ifica \textit{``Science Uncovered'' en el Museo de Historia Natural de Londres}. Presentaci\'on del proyecto PREDICTS \href{www.predicts.org.uk/}{www.predicts.org.uk} y del proyecto de biodiversidad de Solanaceae  \href{www.solanaceaesource.org}{www.solanaceaesource.org}}.

\cvline{2015--2016}{Secretaria, \textit{Sociedad Latinoamericana, Imperial College London}, Reino Unido.}


\vspace{0.25cm}
\section{HABILIDADES EN COMPUTACI\'ON}
\cvline{Programaci\'on}{R (avanzado)}{}{}
\cvline{}{Python (intermedio)}{}{}
\cvline{}{C++, shell, HPC cluster scripts (b\'asico)}{}{}
\cvline{Ciencia de datos}{limpieza, procesamiento, visualizaci\'on, an\'alisis estad\'isticos (avanzado)}
\cvline{}{GIS en R y ArcGIS (intermedio)}{}{}
\cvline{Control de versi\'on}{Git}{}{}
\cvline{OS}{Linux/Unix, OS X, Windows}{}{}
\cvline{Tipograf\'ia}{\LaTeX, OpenOffice/LibreOffice}{}{}
%\cvcomputer{Phylogeny software}{POY, TNT, Nona, Winclada, PhyML, MrBayes, Bioedit, PAUP, R, BAMM}{}{}
%\cvcomputer{Biogeography software}{BioGeoBears, NDM, VNDM, Component 1.5, Component 2, DIVA, TREEFITTER, Martitracks}{}{}


\vspace{0.25cm}
\section{LENGUAJES}

\cvline{Espa\~nol}{Nativa}
\cvline{Ingles}{Nivel Profesional}
\cvline{Italiano}{Nivel B\'asico}


%\section{Undergraduate Thesis}


%\section{Awards	}

%Academic Stimuli. First place (Highest GPA) in # semester. Academic Program of Biology. Universidad Industrial de Santander.


%\newpage



%\section{Postgraduate courses}
%
%\cvline{2013}{Postgraduate Course in C programming}
%\cvline{}{Imperial College London}
%
%\cvline{2009}{Introduction to evolutionary biogeography. Facultad de Ciencias Exactas y Naturales. Posgrado de Biolog\'ia. Universidad de Antioquia. Medell\'in, Colombia.}
%\cvline{supervisors}{Juan Jose Morrone Lupi, Tania Escalante Espinosa}


%\section{Other courses}
%
%\cvline{2011}{Introduction to web based programming (HTML and JAVASCRIPT). Servicio Nacional de Aprendizaje SENA}
%
%\cvline{2011}{Java applet development, events management, classes and objects. Servicio Nacional de Aprendizaje SENA.}
%
%\cvline{2011}{Modules, storage structures and OOP using C++ programming language( Level II). Servicio Nacional de Aprendizaje SENA.}
%
%\cvline{2010}{Structure oriented programming language C++ (Level I). Servicio Nacional de Aprendizaje SENA.}
%
%\cvline{2010}{Variables, and control structures in object oriented programming: JAVA. Servicio Nacional de Aprendizaje SENA.}



%\section{Training courses and workshops}
%
%\cvline{2014}{Plant taxonomy course. NERC-funded training courses. The Natural History Museum, London, United Kingdom}
%
%\cvline{2014}{Molecular techniques for taxonomy. NERC-funded training courses. The Natural History Museum, London, United Kingdom}
%
%\cvline{2014}{Radiation and extinction: Investigating clade dynamics in deep time workshop. The Linnean Society and Imperial College London, London, United Kingdom}
%
%\cvline{2014}{Phylogeny, extinction risks and conservation discussion meeting. The Royal Society, London, United Kingdom}
%
%\cvline{2013}{Spatial analysis in R. BES Macroecology SIG workshop. Sheffield University, Sheffield, United Kingdom}
%
%\cvline{2013}{Postgraduate course in C programming, Imperial College London, London, United Kingdom}
%
%\cvline{2013}{The generalised linear modelling (GLIM) course presented by Prof. M. J. Crawley, Imperial College London, London, United Kingdom}
%
%\cvline{2009}{Introduction to evolutionary biogeography. Facultad de Ciencias Exactas y Naturales. Posgrado de Biolog\'ia. Universidad de Antioquia. Medell\'in, Colombia.}



%\cvline{2015}{International development conference (IDC). Imperial College London, London, United Kingdom}
%
%\cvline{2014}{16th Young Systematists' forum. Natural History Museum, London, United Kingdom}
%
%
%
%\cvline{2014}{Plants Radiation meeting. Institute of Systematic Botany, University of Z\"urich, Switzerland}
%
%\cvline{2013}{11th INTECOL Congress, Ecology, London, United Kingdom}
%
%\cvline{2010}{III Congreso Colombiano de Zoolog\'ia. Medell\'in, Colombia.}
%\cvline{2009}{III Simposio red colombiana de Biolog\'ia Evolutiva.-Colevol-. Cali-Colombia}
%\cvline{2007}{V Congreso Internacional y VIII Congreso Colombiano de Gen\'etica. Cali, Colombia.}
%\cvline{2007}{I Congreso Nacional de Estudiantes de Biolog\'ia. Bogot\'a. Colombia.}
%\cvline{2006}{IV Congreso Internacional y VII Congreso Colombiano de Gen\'etica.  Bucaramanga, Colombia.}
%
%\section{\textbf{Languages}}
%\cvlanguage{2012}{Pre-sessional course in English.}{Level of competence in English: 7.0 \\ University of Bath}

%\cvlanguage{2011}{TOEFL IBT}{
%Score 98/120\\
%Reading score: 29\\
%Listening score: 25\\
%Speaking score: 23\\
%Writing score: 21\\
%ETS ID:	5653379\\
%Registration Number: 0000000013635906\\
%Date of examination: 17-Dec-2011
%}
%\cvlanguage{2010}{Michigan English Test}{Score 127}


\vspace{0.25cm}
\section{REFERENCIAS}

\begin{minipage}[t]{.5\textwidth}
\raggedright
\cvreference{Prof. Andrew J. Kerkhoff}
	{Profesor de Biolog\'ia}
	{Departamento de Biolog\'ia}
	{Kenyon College}
	{Higley Hall, Gambier, OH 43022}
	{kerkhoffa@kenyon.edu}
	{+1 740 427 5734}
\end{minipage}
\hfill
\begin{minipage}[t]{.5\textwidth}
\raggedright

\cvreference{Prof. Andy Purvis}
	{Investigador lider}
	{Departamento de Ciencias de la Vida}
	{Natural History Museum, Imperial College London}
	{Cromwell Rd, Londres SW7 5BD}
	{Andy.Purvis@nhm.ac.uk}
	{+44 020 7942 5686}
\end{minipage}

\vspace{0.5cm}

\begin{minipage}[t]{.5\textwidth}
\raggedright
\cvreference{Dr Sandra Knapp}
	{Investigadora y directora de la divisi\'on de bot\'anica}
    	{Departamento de Ciencias de la Vida}
    	{Natural History Museum}
    	{Cromwell Rd, Londres SW7 5BD}
    	{s.knapp@nhm.ac.uk}
    	{+44 020 7942 5171}
\end{minipage}
\hfill
\begin{minipage}[t]{.5\textwidth}
\raggedright
\cvreference{Dr Natalie Cooper}
	{Investigadora}
    	{Departamento de Ciencias de la Vida}
    	{Natural History Museum}
    	{Cromwell Rd, Londres SW7 5BD}
    	{natalie.cooper@nhm.ac.uk}
    	{+44 020 7942 5083}
\end{minipage}



\closesection{}                   				% needed to renewcommands
\renewcommand{\listitemsymbol}{-} 		% change the symbol for lists

%\section{Extra 1}
%\cvlistitem{Item 1}
%\cvlistitem{Item 2}
%\cvlistitem[+]{Item 3}            % optional other symbol

%\section{Extra 2}
%\cvlistdoubleitem[\Neutral]{Item 1}{Item 4}
%\cvlistdoubleitem[\Neutral]{Item 2}{Item 5}
%cvlistdoubleitem[\Neutral]{Item 3}{}

% Publications from a BibTeX file
%\nocite{*}
%\bibliographystyle{plain}
%\bibliography{publications}       % 'publications' is the name of a BibTeX file

\end{document}


%% end of file `template_en.tex'.
