% !TeX spellcheck = en_GB
% !TeX program = pdflatex
%
% LuxSleek-CV 1.1 LaTeX template
% Author: Andreï V. Kostyrka, University of Luxembourg
%
% 1.1: added tracking and letter-spacing for prettier lower caps, added `~` for language levels
% 1.0: initial release
%
% This template fills the gap in the available variety of templates
% by proposing something that is not a custom class, not using any
% hard-coded settings deeply hidden in style files, and provides
% a handful of custom command definitions that are as transparent as it gets.
% Developed at the University of Luxembourg.
%
% *NOTHING IS HARCODED, and never should be.*
%
% Target audience: applicants in the IT industry, or business in general
%
% The main strength of this template is, it explicitly showcases how
% to break the flow of text to achieve the most flexible right alignment
% of dates for multiple configurations.

\documentclass[10pt, a4paper]{article} 

\usepackage[T1]{fontenc}     % We are using pdfLaTeX,
\usepackage[utf8]{inputenc}  % hence this preparation
\usepackage[british]{babel}  
\usepackage[left = 0mm, right = 0mm, top = 0mm, bottom = 0mm]{geometry}
\usepackage[stretch = 25, shrink = 25, tracking=true, letterspace=30]{microtype}  
\usepackage{graphicx}        % To insert pictures
\usepackage{xcolor}          % To add colour to the document
\usepackage{marvosym}        % Provides icons for the contact details

\usepackage{enumitem}        % To redefine spacing in lists
\setlist{parsep = 0pt, topsep = 0pt, partopsep = 1pt, itemsep = 1pt, leftmargin = 6mm}

\usepackage[default]{FiraSans}     % Change this to use any font, but keep it simple
\renewcommand{\familydefault}{\sfdefault}

\definecolor{cvblue}{HTML}{304263}

%%%%%%% USER COMMAND DEFINITIONS %%%%%%%%%%%%%%%%%%%%%%%%%%%
% These are the real workhorses of this template
\newcommand{\dates}[1]{\hfill\mbox{\textbf{#1}}} % Bold stuff that doesn’t got broken into lines
\newcommand{\is}{\par\vskip.5ex plus .4ex} % Item spacing
\newcommand{\smaller}[1]{{\small$\diamond$\ #1}}
\newcommand{\headleft}[1]{\vspace*{3ex}\textsc{\textbf{#1}}\par%
    \vspace*{-1.5ex}\hrulefill\par\vspace*{0.7ex}}
\newcommand{\headright}[1]{\vspace*{2.5ex}\textsc{\Large\color{cvblue}#1}\par%
     \vspace*{-2ex}{\color{cvblue}\hrulefill}\par}
%%%%%%%%%%%%%%%%%%%%%%%%%%%%%%%%%%%%%%%%%%%%%%%%%%%%%%%%%%%%

\usepackage[colorlinks = true, urlcolor = white, linkcolor = white]{hyperref}

\begin{document}

% Style definitions -- killing the unnecessary space and adding the skips explicitly
\setlength{\topskip}{0pt}
\setlength{\parindent}{0pt}
\setlength{\parskip}{0pt}
\setlength{\fboxsep}{0pt}
\pagestyle{empty}
\raggedbottom

\begin{minipage}[t]{0.33\textwidth} %% Left column -- outer definition
%  Left column -- top dark rectangle
\colorbox{cvblue}{\begin{minipage}[t][5mm][t]{\textwidth}\null\hfill\null\end{minipage}}

\vspace{-.2ex} % Eliminates the small gap
\colorbox{cvblue!90}{\color{white}  %% LEFT BOX
\kern0.09\textwidth\relax% Left margin provided explicitly
\begin{minipage}[t][293mm][t]{0.82\textwidth}
\raggedright
\vspace*{2.5ex}

\Large Susy \textbf{\textsc{Echeverria-Londono, PhD}} \normalsize 

% Centering without extra vertical spacing
%\null\hfill\includegraphics[width=0.65\textwidth]{Profile.png}\hfill\null

\headleft{Coordonnées}
\small % To fit more content
\MVAt\ {\small susyelo@gmail.com} \\[0.4ex]
\Mobilefone\ +33 6 36 01 15 45 \\[0.5ex]
\Mundus\ \href{https://susyelo.github.io/}{susyelo.github.io} \\[0.1ex]
\Letter\ Région parisienne, France
\normalsize

\vspace*{0.5ex} % Extra space after the picture

\headleft{Résumé professionnel}
\small Chercheur pluridisciplinaire avec plus de 10 ans d’expérience à l’intersection de la biodiversité, de la santé publique et des sciences de l’environnement. Expert dans la synthèse et la communication de résultats scientifiques complexes à destination des chercheurs, décideurs politiques et grand public. Compétent dans la gestion et l’organisation d’événements internationaux. Solides compétences en synthèse scientifique, visualisation de données et coordination collaborative à travers l’Europe, les Amériques et des consortiums internationaux (par ex. OMS, GAVI, NSF, VIMC).


\headleft{Compétences}
\normalsize

\begin{itemize}

\item Direction de projets de recherche internationaux et publications scientifiques (par ex. VIMC, financés par NSF)
\item Expérience dans la présentation et la rédaction de rapports pour des publics variés, incluant scientifiques, décideurs et grand public.
\item Solide expérience de travail avec des équipes internationales.
\item Expérience dans la coopération bilatérale et internationale (Amérique latine et Royaume-Uni).
\item Grande capacité à synthétiser des données complexes en informations accessibles.
\item Proactif et bien organisé, avec une forte capacité à prioriser et un bon esprit d’équipe.
\item Multilingue (espagnol natif, anglais professionnel, français/italien intermédiaire).
\item Outils : R, Python, Git, Excel, LaTeX, Suite Office 
\end{itemize} 

\headleft{Centres d’intérêt}
En dehors du travail, je pratique la danse, je me promène régulièrement avec mon chien et ma famille, et je fais du bénévolat dans des refuges pour chiens.



\end{minipage}%
\kern0.09\textwidth\relax%%Right margin provided explicitly to stretch the colourbox
}
\end{minipage}% Right column
\hskip2.5em% Left margin for the white area
\begin{minipage}[t]{0.56\textwidth}
\setlength{\parskip}{0.8ex}% Adds spaces between paragraphs; use \\ to add new lines without this space. Shrink this amount to fit more data vertically

\vspace{2ex}

\headright{Expérience professionnelle}

\is
\textsc{Consultant en recherche (freelance)} \dates{2022-- aujourd’hui}  \\  
\textit{Global / Télétravail}   \\
{\small \begin{itemize}
	\item Soutien aux institutions de politique publique dans l’évaluation du potentiel d’harmonisation et d’intégration des estimations d’impact vaccinal à travers plusieurs plateformes.
  	\item Communication des résultats à travers des rapports destinés à des parties prenantes mondiales (VIMC, GAVI et BMGF).
\end{itemize}}
        		
\is	
\textsc{Consultant et Associé de recherche}    \dates{2019--2022} \\
\textit{Imperial College London, Royaume-Uni} \\
{\small  \begin{itemize}     
  	\item Collaboration avec une équipe multidisciplinaire de chercheurs, d’experts techniques, d’organisations axées sur les politiques et de chefs de projet pour produire des rapports de recherche et d’orientation politique.
  	\item Production de livrables robustes et pertinents pour les politiques de santé mondiale ; rédaction régulière de rapports techniques pour soutenir la prise de décision fondée sur des données probantes.
  	\item Rédaction de notes de synthèse et de rapports pour appuyer les décisions stratégiques en matière de vaccination à l’échelle mondiale.
  	\item Collaboration avec des chercheurs colombiens pour comprendre la propagation et l’évolution initiale du SARS-CoV-2 en Colombie.
\end{itemize}}

\is
\textsc{Chercheur postdoctoral (NSF)}  \dates{2017--2019} \\
\textit{Kenyon College, Ohio, États-Unis} \\
{\small \begin{itemize}
	\item Direction de recherches sur la biodiversité végétale à travers différents biomes en Amérique, intégrant des données phylogénétiques et écologiques.
	\item Co-développement et enseignement d’un module international “Écologie globale et biogéographie”.
 	\item Collaboration avec des chercheurs brésiliens sur l’évolution des biomes. \textit{PIs : Prof. Andrew Kerkhoff et Prof. Brian J. Enquist}
\end{itemize} }

\is  
\textsc{Chargé de cours et démonstrateur} \dates{2013--2017} \\  
\textit{Muséum d’Histoire Naturelle / Imperial College London, Royaume-Uni} \\
{\small \begin{itemize}
  	\item Coordination de la collecte et de l’analyse de données sur le terrain entre chercheurs colombiens et institutions britanniques.
  	\item Contribution à la diplomatie scientifique de haut niveau, présentation des résultats de recherche à des personnalités comme le président colombien, des ministres et le prince Charles.
\end{itemize} }

\headright{Formation}
\small
\textsc{Doctorat (Ph.D.)} en Sciences de la Vie  \dates{2013--2017} \\
 \textit{Imperial College London, Royaume-Uni}.\\
%\smaller{\textit{Thèse : Macrovévolution des plantes et réponses de la biodiversité aux changements d’utilisation des terres.}}

\is
\textsc{Master Recherche (M.Res.)} en Biodiversité, Informatique et Génomique  avec distinction \dates{2012--2013} \\
\textit{Imperial College London, Royaume-Uni}.  \\
%\smaller{Thèse : Effets de l’usage des terres sur la biodiversité locale en Colombie.}

\is
\textsc{Licence en Biologie} (mention très bien)  \dates{2004--2010} \\
\textit{Universidad Industrial de Santander, Colombie}. 

\headright{Projets et Réalisations}
\small
\begin{itemize}
\item Gestion de la livraison hebdomadaire de rapports destinés à l’OMS, GAVI, et BMGF, en respectant des délais stricts.
\item Participation et organisation de réunions internationales de travail ; coordination de jeux de données multi-pays ; appui à la mise en place d’accords de coopération.
\item Coordination de la collecte de données et de la communication entre chercheurs colombiens et européens pour des projets de suivi de biodiversité (ex. : PREDICTS).
\item Plus de 30 articles dans des revues à comité de lecture (Lancet, Nature, eLife, Vaccine) : \url{https://susyelo.github.io/publications.html}.
\item Lauréat du prix "Outstanding Paper by Young Investigator" (JSE, 2022).
\end{itemize}

\end{minipage}

\end{document}