%%%%%%%%%%%%%%%%%%%%%%%%%%%%%%%%%%%%%%%%%%%%%%%%%%%%%%%%%%%%%%%%%%%%%%%%%%%%
%
% Twenty Seconds Curriculum Vitae in LaTex
% ****************************************
% 
% License: MIT
%
% For further information please visit:
% https://github.com/spagnuolocarmine/TwentySecondsCurriculumVitae-LaTex
% 
%%%%%%%%%%%%%%%%%%%%%%%%%%%%%%%%%%%%%%%%%%%%%%%%%%%%%%%%%%%%%%%%%%%%%%%%%%%%


% !TEX none

\documentclass[icon]{twentysecondcv}
\begin{document}


%%%%%%%%%%%%%%%%%%%%%%
%% PROFILE SIDE BAR %%
%%%%%%%%%%%%%%%%%%%%%%
% personal info
\profilepic{Profile.png}          % path of profile pic
\cvname{Susy\\  Echeverr\'ia-Londono}                   % your name
\cvjobtitle{Data scientist spécialisée \\ en biodiversité et santé publique}          % your actual job position
%\cvdate{26 November 1865}        % date of birth
\cvaddress{Région parisienne}       % address
\cvnumberphone{+33 6 36 01 15 45}   % telephone number
\cvmail{susyelo@gmail.com}    % e-mail
\cvsite{https://susyelo.github.io} % personal site

\aboutme{

\noindent Je suis data scientist, spécialisée dans la biodiversité et la santé publique, avec plus de dix ans d'expérience dans la conduite et la publication de travaux de recherche. Mon expertise porte sur l'analyse de grands ensembles de données complexes et \\ l'application de méthodes statistiques pour identifier les tendances globales en santé publique et les interactions complexes entre biodiversité, environnement et activités humaines.

} % about me section

%%%%%%%%%%%%%%%%%%%%%%%%%%%%%%%%%%%%%%%%%%%%%%%%%%%%%%%%%%%%%%%%%%%%%%%%%%%%%%%%%%%%%
%%%%%% Skill bar section, each skill must have a value between 0 an 6 (float) %%%%%%%
%%%%%%%%%%%%%%%%%%%%%%%%%%%%%%%%%%%%%%%%%%%%%%%%%%%%%%%%%%%%%%%%%%%%%%%%%%%%%%%%%%%%%

\skills{{Programmation en R/5},{SIG, $\textbullet$ Modélisation statistique/4.5}, {Git, $\textbullet$\large LaTeX/ 4}, {Python, $\textbullet$ Scripts pour clusters HPC, $\textbullet$ Shell/ 2}, 
{Espagnol/5},{Anglais/4.5}, {Italien/4}, {Français/3.5}}




\makeprofile

%%%%%%%%%%%%%%%%%%%%%%%%%%
%% END PROFILE SIDE BAR %%
%%%%%%%%%%%%%%%%%%%%%%%%%%

%%%%%%%%%%%%%%%%%%
%%%%%% BODY %%%%%%
%%%%%%%%%%%%%%%%%%

%%%%%%%%%%%%%%%%%%%%
%% SIMPLE SECTION %%
%%%%%%%%%%%%%%%%%%%%


\section{Expérience}

\begin{twenty} % Environment for a list with descriptions

% \twentyitem{<dates>}{<title>}{<location>}{<description>}

 \twentyitem
    	{2022-\\présent}
    	{Consultante en recherche}
	{Indépendante}
        	{\\ \small Évaluation et comparaison des estimations d’impact des vaccins contre la dengue, la tuberculose, le streptocoque du groupe B et la shigellose issues de la Stratégie d’investissement dans les vaccins 2024 de GAVI avec celles des vaccins du Vaccine Impact Modelling Consortium.}
        	
	
 \twentyitem
       {2019-2022}
       {Consultante en recherche \\ et associée de recherche}
       {Imperial College London, R.-U.}
       {\\ \small Analyse, organisation et préparation des estimations d’impact vaccinal pour 12 pathogènes (choléra, hépatite B, Hib, HPV, encéphalite japonaise, rougeole, MenA, PCV, rotavirus, rubéole, typhoïde et fièvre jaune) dans 112 pays, couvrant la période de 2000 à 2030. \textit{Chef de projet : Prof. Neil M. Ferguson, responsable : Dr Katy Gaythorpe}.}
      
\twentyitem
    	{2018- 2019}
	{Chercheuse invitée}
        	{University of Pittsburgh, E.-U.}
        	{\\ \small Évaluation et prédiction des risques d’extinction de 300 espèces végétales grâce à une exploration approfondie et une analyse de leurs données spatio-temporelles, en utilisant l’analyse de schéma ponctuel spatial (spatial point pattern analysis).  \textit{Chef de projet : Dr Justin Kitzes}.}

	\twentyitem
    	{2017-2019}
	{Associée postdoctorale NSF}
        {Kenyon College, E.-U.}
        {\\ \small Analyse approfondie des distributions de diversité fonctionnelle parmi les espèces végétales à travers les biomes d’Amérique du Nord et du Sud. Utilisation des données de distribution d’environ 85 000 espèces végétales, comprenant près de 9 millions de points géographiques, en les croisant avec des traits fonctionnels disponibles publiquement. \textit{Chef de projet : Prof. Andrew Kerkhoff et  Prof. Brian J. Enquist}. }
      
    
    \twentyitem
	{2013-2017}
	{Chargée de cours et travaux dirigés}
	{Natural History Museum, Imperial College London, R.-U.}
	{}
	%{\\ \small Enseigné et réalisé des démonstrations pour plusieurs cours quantitatifs dans les programmes de master en Taxonomie et Biodiversité et de licence en Sciences de la vie.}
	%
\end{twenty}


\section{Éducation}

\begin{twenty}
  \twentyitem
    {2013-2017}
    {Doctorat  {\normalfont en sciences de la vie}}
    {Imperial College London, R.-U.}
    {}
  \twentyitem
    {2012-2013}
    {M.Res. avec mention très bien \\  {\normalfont en biodiversité, informatique et génomique}}
    {Imperial College London, R.-U.}
    {}
  \twentyitem
    {2004-2010}
    {Licence en biologie  \\  mention très bien}
    {Universidad Industrial de Santander, Colombie}
    {}
 \end{twenty}
 
 
\section{Publications sélectionnées}

\small 2022 \textit{Echeverr\'ia-Londo\~no, S,.}  Hartner, A. M., Li, X., Roth, J., Portnoy, A., Sbarra, A. N., ... \& Gaythorpe, K. A. Exploring the subnational inequality and heterogeneity of the impact of routine measles immunisation in Africa. \textit{Vaccine}, 40(47), 6806-6817. 

\small  2021 \textit{Echeverr\'ia-Londo\~no, S,.}  Li, X., Toor, J., de-Villiers, M., Nayagam, S., Hallett, T.B., Abbas, K., Jit, M., Klepac, P., Jean, K. \&  Garske, T. How can the public health impact of vaccination be estimated? \textit{BMC Public Health}, 21, 2049 (2021).

\small  2020. \textit{Echeverr\'ia-Londo\~no, S,.}  S{\"a}rkinen, T., Fenton, I. S., Knapp, S. and Purvis, A. Dynamism and context dependency in the diversification of the megadiverse plant genus Solanum L. (Solanaceae), \textit{Journal of Systematics and Evolution},  58(6), 767-782. 

\small  2018. \textit{Echeverr\'ia-Londo\~no, S.,} Enquist. B. J., Neves, D. M., Violle, C. and Kerkhoff, A. J. Plant functional diversity and the biogeography of biomes in North and South America. \textit{Frontiers in Ecology and Evolution}, 6(DEC), 219.

\small  2016. \textit{Echeverr\'ia-Londo\~no, S.,} Newbold, T., Hudson, L. N., Hill, S. L., Contu, S., Lysenko, I., \dots and Purvis, A. Modelling and projecting the response of Colombian biodiversity to land-use change, \textit{Diversity and Distributions}, 22, 1099--1111. 
%%%%%%%%%%%%%%%%%%%%%%
%%%%% ENDBODY %%%%%%%%
%%%%%%%%%%%%%%%%%%%%%%
\end{document} 