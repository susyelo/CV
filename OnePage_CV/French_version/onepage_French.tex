%%%%%%%%%%%%%%%%%%%%%%%%%%%%%%%%%%%%%%%%%%%%%%%%%%%%%%%%%%%%%%%%%%%%%%%%%%%%
%
% Twenty Seconds Curriculum Vitae in LaTex
% ****************************************
% 
% License: MIT
%
% For further information please visit:
% https://github.com/spagnuolocarmine/TwentySecondsCurriculumVitae-LaTex
% 
%%%%%%%%%%%%%%%%%%%%%%%%%%%%%%%%%%%%%%%%%%%%%%%%%%%%%%%%%%%%%%%%%%%%%%%%%%%%


% !TEX none

\documentclass[icon]{twentysecondcv}
\begin{document}


%%%%%%%%%%%%%%%%%%%%%%
%% PROFILE SIDE BAR %%
%%%%%%%%%%%%%%%%%%%%%%
% personal info
\profilepic{Profile.png}          % path of profile pic
\cvname{Susy\\  Echeverr\'ia-Londono, PhD}                   % your name
\cvjobtitle{Chercheur et data scientist}          % your actual job position
%\cvdate{26 November 1865}        % date of birth
\cvaddress{Région parisienne}       % address
\cvnumberphone{+33 6 36 01 15 45}   % telephone number
\cvmail{susyelo@gmail.com}    % e-mail
\cvsite{https://susyelo.github.io} % personal site

\aboutme{

\noindent Je suis chercheur et data scientist, spécialisé dans les projets de recherche axés sur les politiques en santé publique et en biodiversité. J’apporte une expertise en science des données, épidémiologie, écologie et évolution, avec une maîtrise de \\  l’analyse de données complexes, statistiques et spatiales pour explorer les tendances mondiales en santé et les liens entre biodiversité, environnement et impact humain.

} % about me section

%%%%%%%%%%%%%%%%%%%%%%%%%%%%%%%%%%%%%%%%%%%%%%%%%%%%%%%%%%%%%%%%%%%%%%%%%%%%%%%%%%%%%
%%%%%% Skill bar section, each skill must have a value between 0 an 6 (float) %%%%%%%
%%%%%%%%%%%%%%%%%%%%%%%%%%%%%%%%%%%%%%%%%%%%%%%%%%%%%%%%%%%%%%%%%%%%%%%%%%%%%%%%%%%%%

\skills{{Programmation en R/5},{SIG, $\textbullet$ Modélisation statistique/4.5}, {Python, Git, $\textbullet$\large LaTeX/ 4}, 
{Espagnol (Langue maternelle)/5},{Anglais (Compétence professionnelle)/4.5}, {Français, Italien (Intermédiaire)/3}}



\makeprofile

%%%%%%%%%%%%%%%%%%%%%%%%%%
%% END PROFILE SIDE BAR %%
%%%%%%%%%%%%%%%%%%%%%%%%%%

%%%%%%%%%%%%%%%%%%
%%%%%% BODY %%%%%%
%%%%%%%%%%%%%%%%%%

%%%%%%%%%%%%%%%%%%%%
%% SIMPLE SECTION %%
%%%%%%%%%%%%%%%%%%%%


\section{Expérience}

\begin{twenty} % Environment for a list with descriptions

% \twentyitem{<dates>}{<title>}{<location>}{<description>}

 \twentyitem
    	{2022-\\présent}
    	{Consultante en recherche}
	{Indépendante}
        	{ \begin{itemize}  \small
        	 \item Analysé et comparé les estimations d’impact vaccinal (dengue, tuberculose, streptocoque B, shigellose) entre GAVI 2024 et le VIMC.
        	 \item Contribué aux efforts de plaidoyer mondial en faveur de la vaccination, en appui aux décideurs politiques.
        	   \end{itemize}
        	
        	}
        	%\\ \small Réalisé une analyse comparative des estimations d’impact des vaccins contre la dengue, la tuberculose, le streptocoque du groupe B et la shigellose entre la Stratégie d’Investissement Vaccinal 2024 de GAVI et celles modélisées par le Vaccine Impact Modelling Consortium (VIMC), afin de soutenir les institutions décisionnelles dans l’exploration du potentiel d’harmonisation et d’intégration des estimations d’impact entre différentes plateformes, en vue de renforcer les efforts mondiaux de plaidoyer pour la vaccination.}
        	
	
 \twentyitem
       {2019-2022}
       {Consultante en recherche \\ et associée de recherche}
       {Imperial College London, R.-U.}
       { \begin{itemize}  \small
       
      \item Travaillé en collaboration avec des chercheurs, experts techniques et acteurs politiques pour assurer la pertinence des résultats.
      \item  Préparé et analysé des séries temporelles d’impact vaccinal issues de 21 modèles développés à l’échelle internationale (112 pays, 12 pathogènes)
      \item Produit des rapports techniques pour orienter les décisions en santé mondiale et soutenir des stratégies d’investissement fondées sur les données.
      \item \textit{Chef de projet : Prof. Neil M. Ferguson, responsable : Dr Katy Gaythorpe}
       \end{itemize}
       }
       
       
       %\\ \small  Analysé, structuré et préparé des estimations d’impact vaccinal en séries temporelles pour 12 agents pathogènes dans 112 pays, en intégrant les résultats de 21 modèles développés par des institutions internationales. Collaboré étroitement avec une équipe multidisciplinaire de chercheurs, d’experts techniques, d’organisations orientées vers les politiques publiques et de chefs de projet afin de garantir des résultats solides et pertinents pour l’élaboration de politiques. Rédigé des rapports techniques réguliers pour informer les parties prenantes et décideurs en santé mondiale, soutenant ainsi des stratégies d’investissement et de plaidoyer vaccinal fondées sur des données probantes. \textit{Chef de projet : Prof. Neil M. Ferguson, responsable : Dr Katy Gaythorpe}.}
 
 
 \twentyitem
    	{2017-2019}
	{Associée postdoctorale NSF}
        {Kenyon College, E.-U.}
        { \begin{itemize}  \small
       
     \item Démontré une maîtrise avancée des données écologiques à grande échelle et des outils d’analyse spatiale.
     \item Mené une analyse quantitative sur ~85 000 espèces végétales et 9 millions de points géoréférencés à travers les biomes d’Amérique.
     \item Intégré des traits fonctionnels publics pour modéliser les schémas de biodiversité. \textit{Chef de projet : Prof. Andrew Kerkhoff et  Prof. Brian J. Enquist}.
     
     \end{itemize}
     }
    
        %\\ \small Mené une analyse quantitative de la distribution de la diversité fonctionnelle parmi les espèces végétales à travers les biomes d’Amérique du Nord et du Sud, en utilisant un ensemble de données d’environ 85 000 espèces végétales et plus de 9 millions de points géographiques géoréférencés. J’ai intégré des traits fonctionnels accessibles au public pour évaluer et modéliser les schémas de biodiversité, démontrant une expertise avancée en analyse de données spatiales et en grands ensembles de données écologiques. \textit{Chef de projet : Prof. Andrew Kerkhoff et  Prof. Brian J. Enquist}. }
        
             
\twentyitem
    	{2018- 2019}
	{Chercheuse invitée}
        	{University of Pittsburgh, E.-U.}
        	{
        	\begin{itemize}  \small
      \item Analysé la distribution spatiale temporelle de 300 espèces végétales à l’aide de méthodes de points ponctuels.
       \item Évalué les risques d’extinction en identifiant des schémas écologiques spatio-temporels. \textit{Chef de projet : Dr Justin Kitzes}.
     \end{itemize}
     
     }
        	
        	%\\ \small Réalisé une analyse spatiale de données sur 300 espèces végétales en appliquant une analyse de distribution spatiale ponctuelle à des données de séries temporelles, afin d’évaluer et de prédire les risques d’extinction à travers l’exploration de schémas écologiques spatiaux et temporels.  \textit{Chef de projet : Dr Justin Kitzes}.}

	    
    
%    \twentyitem
%	{2013-2017}
%	{Chargée de cours et travaux dirigés}
%	{Natural History Museum, Imperial College London, R.-U.}
	%{}
	%{\\ \small Enseigné et réalisé des démonstrations pour plusieurs cours quantitatifs dans les programmes de master en Taxonomie et Biodiversité et de licence en Sciences de la vie.}
	%
\end{twenty}


\section{Éducation}

\begin{twenty}
  \twentyitem
    {2013-2017}
    {Doctorat  {\normalfont en sciences de la vie}}
    {Imperial College London, R.-U.}
    {}
  \twentyitem
    {2012-2013}
    {M.Res. avec mention très bien \\  {\normalfont en biodiversité, informatique et génomique}}
    {Imperial College London, R.-U.}
    {}
  \twentyitem
    {2004-2010}
    {Licence en biologie  \\  mention très bien}
    {Universidad Industrial de Santander, Colombie}
    {}
 \end{twenty}
 
 
\section{Publications sélectionnées}

\small 2022 \textit{Echeverr\'ia-Londo\~no, S,.}  Hartner, A. M., Li, X., Roth, J., Portnoy, A., Sbarra, A. N., ... \& Gaythorpe, K. A. Exploring the subnational inequality and heterogeneity of the impact of routine measles immunisation in Africa. \textit{Vaccine}, 40(47), 6806-6817. 

\small  2021 \textit{Echeverr\'ia-Londo\~no, S,.}  Li, X., Toor, J., de-Villiers, M., Nayagam, S., Hallett, T.B., Abbas, K., Jit, M., Klepac, P., Jean, K. \&  Garske, T. How can the public health impact of vaccination be estimated? \textit{BMC Public Health}, 21, 2049 (2021).

\small  2020. \textit{Echeverr\'ia-Londo\~no, S,.}  S{\"a}rkinen, T., Fenton, I. S., Knapp, S. and Purvis, A. Dynamism and context dependency in the diversification of the megadiverse plant genus Solanum L. (Solanaceae), \textit{Journal of Systematics and Evolution},  58(6), 767-782. 

\small  2018. \textit{Echeverr\'ia-Londo\~no, S.,} Enquist. B. J., Neves, D. M., Violle, C. and Kerkhoff, A. J. Plant functional diversity and the biogeography of biomes in North and South America. \textit{Frontiers in Ecology and Evolution}, 6(DEC), 219.

\small  2016. \textit{Echeverr\'ia-Londo\~no, S.,} Newbold, T., Hudson, L. N., Hill, S. L., Contu, S., Lysenko, I., \dots and Purvis, A. Modelling and projecting the response of Colombian biodiversity to land-use change, \textit{Diversity and Distributions}, 22, 1099--1111. 
%%%%%%%%%%%%%%%%%%%%%%
%%%%% ENDBODY %%%%%%%%
%%%%%%%%%%%%%%%%%%%%%%
\end{document} 