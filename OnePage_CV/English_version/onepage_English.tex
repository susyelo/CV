%%%%%%%%%%%%%%%%%%%%%%%%%%%%%%%%%%%%%%%%%%%%%%%%%%%%%%%%%%%%%%%%%%%%%%%%%%%%
%
% Twenty Seconds Curriculum Vitae in LaTex
% ****************************************
% 
% License: MIT
%
% For further information please visit:
% https://github.com/spagnuolocarmine/TwentySecondsCurriculumVitae-LaTex
% 
%%%%%%%%%%%%%%%%%%%%%%%%%%%%%%%%%%%%%%%%%%%%%%%%%%%%%%%%%%%%%%%%%%%%%%%%%%%%

\documentclass[icon]{twentysecondcv}
\begin{document}

%%%%%%%%%%%%%%%%%%%%%%
%% PROFILE SIDE BAR %%
%%%%%%%%%%%%%%%%%%%%%%
% personal info
\profilepic{Profile.png}          % path of profile pic
\cvname{Susy\\  Echeverr\'ia-Londono, PhD}                   % your name
\cvjobtitle{Researcher and data scientists}          % your actual job position
%\cvdate{26 November 1865}        % date of birth
\cvaddress{Paris area, France}       % address
\cvnumberphone{+33 6 36 01 15 45}   % telephone number
\cvmail{susyelo@gmail.com}    % e-mail
\cvsite{https://susyelo.github.io} % personal site

\aboutme{

\noindent I am a researcher and data scientist specialising in policy oriented research projects in public health and biodiversity. I have successfully contributed to international organisation and research teams combining expertise in data science in the areas of public health, epidemiology, ecology and evolution. My expertise lies in analysing large, complex datasets and leveraging statistical approaches and spatial analysis to uncover global patterns in public health coverage and the intricate relationships between biodiversity, environment, and human impact. 

} % about me section

%%%%%%%%%%%%%%%%%%%%%%%%%%%%%%%%%%%%%%%%%%%%%%%%%%%%%%%%%%%%%%%%%%%%%%%%%%%%%%%%%%%%%
%%%%%% Skill bar section, each skill must have a value between 0 an 6 (float) %%%%%%%
%%%%%%%%%%%%%%%%%%%%%%%%%%%%%%%%%%%%%%%%%%%%%%%%%%%%%%%%%%%%%%%%%%%%%%%%%%%%%%%%%%%%%
\skills{{R Programming/5},{GIS, $\textbullet$ Statistics/4.5}, {Python, Git, $\textbullet$\large LaTeX/ 4}, 
 Spanish (Native)/5,
  English (Professional Proficiency)/5,
  French (Intermediate)/3,
  Italian (Intermediate)/3}


\makeprofile

%%%%%%%%%%%%%%%%%%%%%%%%%%
%% END PROFILE SIDE BAR %%
%%%%%%%%%%%%%%%%%%%%%%%%%%

%%%%%%%%%%%%%%%%%%
%%%%%% BODY %%%%%%
%%%%%%%%%%%%%%%%%%

%%%%%%%%%%%%%%%%%%%%
%% SIMPLE SECTION %%
%%%%%%%%%%%%%%%%%%%%


\section{Experience}

\begin{twenty} % Environment for a list with descriptions

% \twentyitem{<dates>}{<title>}{<location>}{<description>}

 \twentyitem
    	{2022-- now}
    	{Research Consultant}
	{Freelance}
        	{\\ \small  Conducted a comparative analysis of vaccine impact estimates for dengue, tuberculosis, group B streptococcus, and shigella from GAVI’s Vaccine Investment Strategy 2024 against those modeled by the Vaccine Impact Modelling Consortium (VIMC), to support policy-making institutions in exploring the potential for harmonizing and integrating impact estimates across platforms to strengthen global vaccine advocacy efforts. }
        	
	
 \twentyitem
       {2019--2022}
       {Research Consultant and Research Associate}
       {Imperial College London, UK}
       { \\ \small Analysed, organised, and prepared time series vaccine impact estimates for 12 pathogens across 112 countries, integrating outputs from 21 models developed by institutions worldwide. Worked closely with a multidisciplinary team of researchers, technical experts, policy-oriented organisations, and project managers to ensure robust and policy-relevant outputs. Produced regular technical reports to inform and update global health stakeholders and decision-makers, supporting evidence-based vaccine investment and advocacy strategies. \textit{ PI: Prof. Neil M. Ferguson and Line manager: Dr Katy Gaythorpe}}
      
\twentyitem
    	{2017--2019}
	{NSF Postdoctoral Associate}
        {Kenyon College, OH, USA}
        {\\  \small Conducted in-depth quantitative analysis of functional diversity distributions among plant species across North and South American biomes, utilising a dataset of approximately 85,000 plant species and over 9 million georeferenced geographic data points. Integrated publicly available functional traits to assess and model biodiversity patterns, demonstrating advanced skills in spatial data analysis and large-scale ecological datasets. \textit{PI: Prof. Andrew Kerkhoff and Prof. Brian J. Enquist} }

\twentyitem
    	{2018--2019}
	{Visitor scholar}
        	{University of Pittsburgh, PA, USA}
        	{\\ \small Conducted spatial data analysis on 300 plant species by applying spatial point pattern analysis to time-series distribution data, assessing and predicting extinction risks through the exploration of spatial and temporal ecological patterns. \textit{PI: Dr. Justin Kitzes}}      
    
 %   \twentyitem
%	{2013--2017}
%	{Lecturer and demonstrator}
%	{Natural History Museum  Imperial College London, UK}
%	{\\ \small Lectured and demonstrated for several quantitative courses in the Taxonomy and Biodiversity MSc and Life Sciences BSc programmes}
	%

\end{twenty}


\section{Education}

\begin{twenty}
  \twentyitem
    {2013--2017}
    {Ph.D. {\normalfont in Life Science}}
    {Imperial College London, UK}
    {\emph{Plant Macroevolution and \\  biodiversity responses to land-use change}}
  \twentyitem
    {2012--2013}
    {M.Res. with Distinction \\  {\normalfont in Biodiversity, Informatics and Genomics}}
    {Imperial College London, UK}
    {Effects of land use  \\  on local biodiversity in Colombia}
  \twentyitem
    {2004--2010}
    {B.Sc. in Biology}
    {Universidad Industrial de Santander, Colombia}
    {1st-class honours}
 \end{twenty}
 
 
 \section{Selected publications}

2022 \textit{Echeverr\'ia-Londo\~no, S,.}  Hartner, A. M., Li, X., Roth, J., Portnoy, A., Sbarra, A. N., ... \& Gaythorpe, K. A. Exploring the subnational inequality and heterogeneity of the impact of routine measles immunisation in Africa. \textit{Vaccine}, 40(47), 6806-6817. 

2021 \textit{Echeverr\'ia-Londo\~no, S,.}  Li, X., Toor, J., de-Villiers, M., Nayagam, S., Hallett, T.B., Abbas, K., Jit, M., Klepac, P., Jean, K. \&  Garske, T. How can the public health impact of vaccination be estimated? \textit{BMC Public Health}, 21, 2049 (2021).

2020. \textit{Echeverr\'ia-Londo\~no, S,.}  S{\"a}rkinen, T., Fenton, I. S., Knapp, S. and Purvis, A. Dynamism and context dependency in the diversification of the megadiverse plant genus Solanum L. (Solanaceae), \textit{Journal of Systematics and Evolution},  58(6), 767-782. 

2018. \textit{Echeverr\'ia-Londo\~no, S.,} Enquist. B. J., Neves, D. M., Violle, C. and Kerkhoff, A. J. Plant functional diversity and the biogeography of biomes in North and South America. \textit{Frontiers in Ecology and Evolution}, 6(DEC), 219.

%2016. \textit{Echeverr\'ia-Londo\~no, S.,} Newbold, T., Hudson, L. N., Hill, S. L., Contu, S., Lysenko, I., \dots and Purvis, A. Modelling and projecting the response of Colombian biodiversity to land-use change, \textit{Diversity and Distributions}, 22, 1099--1111. 


%%%%%%%%%%%%%%%%%%%%%%
%%%%% ENDBODY %%%%%%%%
%%%%%%%%%%%%%%%%%%%%%%

\end{document} 