%%%%%%%%%%%%%%%%%%%%%%%%%%%%%%%%%%%%%%%%%
% Twenty Seconds Resume/CV
% LaTeX Template
% Version 1.0 (14/7/16)
%
% Original author:
% Carmine Spagnuolo (cspagnuolo@unisa.it) with major modifications by 
% Vel (vel@LaTeXTemplates.com) and Harsh (harsh.gadgil@gmail.com)
%
% License:
% The MIT License (see included LICENSE file)
%
%%%%%%%%%%%%%%%%%%%%%%%%%%%%%%%%%%%%%%%%%

%----------------------------------------------------------------------------------------
%	PACKAGES AND OTHER DOCUMENT CONFIGURATIONS
%----------------------------------------------------------------------------------------

\documentclass[ a4paper]{twentysecondcv} % a4paper for A4

\usepackage{scrextend}
\changefontsizes{9pt}



% Projects text
\education{
	\textbf{PhD in Life Science} \\
	Imperial College London
	\\
	The Natural History Museum \\
	2013 - 2017 | London, UK. 
	
	\textbf{MRes Biodiversity, Informatics and \\ Genomics} (Distinction) \\
	Imperial College London \\
	2012 - 2013 | Silwood Park, Ascot, UK.
	
	\textbf{BSc in Biology} (1st class honours) \\
	Universidad Industrial de Santander \\
	2004 - 2010 | Bucaramanga, Colombia.
}

% Command for printing skill overview bubbles
\newcommand\skills{ 
	~
	\smartdiagram[bubble diagram]{
		\textbf{~~~~Biodiversity~~~~}\\\textbf{~~~~Public Health~~~~},
		\textbf{Data}\\\textbf{processing},
		\textbf{~~~~Statistics~~~~},
		\textbf{Reproducibility},
		\textbf{Visualisations}
	}
}

% Programming skill bars
\programming{{Python $\textbullet$ HPC cluster scripts $\textbullet$ Shell/ 2}, {git $\textbullet$\large \LaTeX/ 3.5},{GIS $\textbullet$ Advanced statistics/ 4}, {R/ 5}}



%----------------------------------------------------------------------------------------
%	 PERSONAL INFORMATION
%----------------------------------------------------------------------------------------
% If you don't need one or more of the below, just remove the content leaving the command, e.g. \cvnumberphone{}

\cvname{Susy \\ Echeverr\'ia-L} % Your name
\cvjobtitle{Biodiversity and Public Health
Research Consultant} % Job
% title/career

\cvlinkedin{}
\cvgithub{https://github.com/susyelo}
\cvnumberphone{+33636011545} % Phone number
\cvsite{https://susyelo.github.io/} % Personal website
\cvmail{susyelo@gmail.com} % Email address
\cvorcid{0000-0002-0038-146X}

%----------------------------------------------------------------------------------------

\begin{document}

\makeprofile % Print the sidebar

%----------------------------------------------------------------------------------------
%	 EXPERIENCE
%----------------------------------------------------------------------------------------

\section{Experience}

\begin{twenty} % Environment for a list with descriptions

 \twentyitem
    	{2024}
		{}
        {Research Consultant}
	{\href{https://www.vaccineimpact.org/}{Vaccine Impact Modelling Consortium (VIMC), Imperial College London, UK}{ PI: Prof. Caroline Trotter  and Line manager: Dr Katy Gaythorpe}}
         {}
         {\begin{itemize}
          \item Collaborated with the science and policy team at VIMC to analyse and incorporate the impact estimates of various vaccines assessed in GAVI's Vaccine Investment Strategy 2024.
         \end{itemize}}
         \\

 \twentyitem
    	{2019--2022}
		{}
        {Research Consultant and Research Associate}
        {\href{https://www.vaccineimpact.org/}{Vaccine Impact Modelling Consortium (VIMC), Imperial College London, UK}{ PI: Prof. Neil M. Ferguson and Line manager: Dr Katy Gaythorpe}}
         {}
         {\begin{itemize}
	\item Conducted thorough analysis, organisation, and preparation of vaccine impact estimates for 12 pathogens (Cholera, HepB, Hib, HPV, Japanese encephalitis, Measles, MenA, PCV, Rotavirus, Rubella, Typhoid, Yellow fever) across 112 countries spanning from 2000 to 2030, incorporating multiple estimates from diverse modelling groups for each pathogen.
	%\item Extracted and standardised subnational vaccine coverage estimates to project the impact of routine measles vaccination across 45 African countries while investigating subnational disparities within these regions from 2000 to 2019.  
	\item Developed methods for aggregation and analysis of disease burden and vaccine impact estimates derived from various epidemiological models.v
	%\item Examined the effects of the COVID-19 pandemic on immunisation activities for 14 pathogens (Diphtheria, HPV, HepB, HIB, Japanese encephalitis, Measles, MenA, PCV, Pertussis, Rotavirus, Rubella, Tetanus, Typhoid, Yellow fever). 
	\end{itemize}}
         \\


\twentyitem
    	{2018-- 2019}
		{}
        {Visitor scholar}
        {\href{https://www.biology.pitt.edu/}{University of Pittsburgh, PA, USA,  PI: Dr. Justin Kitzes}}
        {}
        {\begin{itemize}
           \item Assessed and predicted extinction risks for 300 plant species through comprehensive exploration and analysis of their spatial time-series data using spatial point pattern analysis
        \end{itemize}}
        \\


	\twentyitem
    	{2017--2019}
		{}
        {NSF Postdoctoral Associate}
        {\href{https://www.kenyon.edu}{Kenyon College, OH, USA, PI: Prof. Andrew Kerkhoff and Prof. Brian J. Enquist}}
        {}
        {
        {\begin{itemize}
     	\item Conducted an extensive analysis of functional diversity distributions among plant species across the biomes of North and South America. Used distribution data from approximately 85,000 plant species, comprising around 9 million geographic points, in conjunction with publicly available functional traits.
	\item Investigated the influence of habitat stability on current patterns of plant diversity analysing the distribution and phylogenetic relationships among approximately 24,000 plant species.
	\item Co-taught the BSc course ``Global Ecology and Biogeography''.
 	\end{itemize}}
        }
    \\   

    \twentyitem
	{2013--2017}
	{}
	{Lecturer \& demonstrator}
	{{Natural History Museum \& Imperial College London, UK}}
	{}
	{
	 {\begin{itemize}
		\item Lectured and demonstrated for several quantitative courses in the Taxonomy and Biodiversity MSc and Life Sciences BSc programmes.
	\end{itemize}}
    	}
        
%	%\twentyitem{<dates>}{<title>}{<location>}{<description>}
\end{twenty}

%----------------------------------------------------------------------------------------
%	 Publications
%----------------------------------------------------------------------------------------

\section{Selected publications}

2022 \textbf{Echeverr\'ia-Londo\~no, S,.}  Hartner, A. M., Li, X., Roth, J., Portnoy, A., Sbarra, A. N., ... \& Gaythorpe, K. A. Exploring the subnational inequality and heterogeneity of the impact of routine measles immunisation in Africa. \textit{Vaccine}, 40(47), 6806-6817. 

2021 \textbf{Echeverr\'ia-Londo\~no, S,.}  Li, X., Toor, J., de-Villiers, M., Nayagam, S., Hallett, T.B., Abbas, K., Jit, M., Klepac, P., Jean, K. \&  Garske, T. How can the public health impact of vaccination be estimated? \textit{BMC Public Health}, 21, 2049 (2021).

2020. \textbf{Echeverr\'ia-Londo\~no, S,.}  S{\"a}rkinen, T., Fenton, I. S., Knapp, S. and Purvis, A. Dynamism and context dependency in the diversification of the megadiverse plant genus Solanum L. (Solanaceae), \textit{Journal of Systematics and Evolution},  58(6), 767-782. 

2018. \textbf{Echeverr\'ia-Londo\~no, S.,} Enquist. B. J., Neves, D. M., Violle, C. and Kerkhoff, A. J. Plant functional diversity and the biogeography of biomes in North and South America. \textit{Frontiers in Ecology and Evolution}, 6(DEC), 219.

2016. \textbf{Echeverr\'ia-Londo\~no, S.,} Newbold, T., Hudson, L. N., Hill, S. L., Contu, S., Lysenko, I., \dots and Purvis, A. Modelling and projecting the response of Colombian biodiversity to land-use change, \textit{Diversity and Distributions}, 22, 1099--1111. 

\end{document}