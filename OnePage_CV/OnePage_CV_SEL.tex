%%%%%%%%%%%%%%%%%%%%%%%%%%%%%%%%%%%%%%%%%
% Twenty Seconds Resume/CV
% LaTeX Template
% Version 1.0 (14/7/16)
%
% Original author:
% Carmine Spagnuolo (cspagnuolo@unisa.it) with major modifications by 
% Vel (vel@LaTeXTemplates.com) and Harsh (harsh.gadgil@gmail.com)
%
% License:
% The MIT License (see included LICENSE file)
%
%%%%%%%%%%%%%%%%%%%%%%%%%%%%%%%%%%%%%%%%%

%----------------------------------------------------------------------------------------
%	PACKAGES AND OTHER DOCUMENT CONFIGURATIONS
%----------------------------------------------------------------------------------------

\documentclass[icon]{twentysecondcv} % a4paper for A4


% Projects text
\education{
	\textbf{PhD in Life Science} \\
	Imperial College London
	\\
	The Natural History Museum \\
	2013 - 2017 | London, UK. 
	
	\textbf{MRes Biodiversity, Informatics and \\ Genomics} (Distinction) \\
	Imperial College London \\
	2012 - 2013 | Silwood Park, Ascot, UK.
	
	\textbf{BSc in Biology} (1st class honours) \\
	Universidad Industrial de Santander \\
	2004 - 2010 | Bucaramanga, Colombia.
}

% Command for printing skill overview bubbles
\newcommand\skills{ 
	~
	\smartdiagram[bubble diagram]{
		\textbf{~~~~Data~~~~}\\\textbf{~~~~Science~~~~},
		\textbf{Data}\\\textbf{processing},
		\textbf{~~~~Statistics~~~~},
		\textbf{Reproducibility},
		\textbf{Visualisations}
	}
}

% Programming skill bars
\programming{{Python $\textbullet$ HPC cluster scripts $\textbullet$ Shell/ 2}, {git $\textbullet$\large \LaTeX/ 3.5},{GIS $\textbullet$ Advanced statistics/ 4}, {R/ 5}}



%----------------------------------------------------------------------------------------
%	 PERSONAL INFORMATION
%----------------------------------------------------------------------------------------
% If you don't need one or more of the below, just remove the content leaving the command, e.g. \cvnumberphone{}

\cvname{Susy \\ Echeverr\'ia-L} % Your name
\cvjobtitle{Biodiversity and Public Health
Research Consultant} % Job
% title/career

\cvlinkedin{}
\cvgithub{https://github.com/susyelo}
\cvnumberphone{+33636011545} % Phone number
\cvsite{https://susyelo.github.io/} % Personal website
\cvmail{susyelo@gmail.com} % Email address
\cvorcid{0000-0002-0038-146X}

%----------------------------------------------------------------------------------------

\begin{document}

\makeprofile % Print the sidebar

%----------------------------------------------------------------------------------------
%	 EXPERIENCE
%----------------------------------------------------------------------------------------

\section{Experience}

\begin{twenty} % Environment for a list with descriptions

 \twentyitem
    	{2024 - }
		{Present}
        {Research Consultant}
        {\href{https://www.vaccineimpact.org/}{Imperial College London, UK}}
         {}
         {\begin{itemize}
         \item Vaccine Impact Modelling Consortium (VIMC). PI: Prof.Caroline Trotter and Line manager: Dr Katy Gaythorpe
         \item Collaborated with the science and policy team at VIMC to analyse and incorporate the impact estimates of various vaccines assessed in the GAVI's Vaccine Investment Strategy 2024.
         \end{itemize}}
         \\
	\twentyitem
    	{2017 -}
		{Present}
        {NSF Postdoctoral Associate}
        {\href{https://www.kenyon.edu}{Kenyon College, OH, USA}}
        {}
        {
        {\begin{itemize}
        \item Explore and analyse the distribution of ca. 85000 plant species including approximately 9 million geographic points. 
        \item Research the consequences of habitat stability on current patterns of plant diversity using the distribution and phylogenetic relationships of ca. 24000 plant species. 
        \item Carry out paleohabitat reconstructions using machine learning methods
        \item Co-teaching of the BSc course ``Global Ecology and Biogeography'' 
        \item Teach and organise lectures and assignments for the Ecoinformatics course which include topics such as: 
        Introduction to data science R, data managing and processing, data visualization, spatial analysis and reproducibility \url{https://globalecologybiogeography.github.io/Ecoinformatics/}		
    \end{itemize}}
        }
    \\   



\twentyitem
    	{2018 -}
		{Present}
        {Visitor scholar}
        {\href{https://www.biology.pitt.edu/}{University of Pittsburgh, PA, USA}}
        {}
        {\begin{itemize}
        \item PI: Dr. Justin Kitzes
        \item Explore and analyse time-series spatial points patterns from ca. 300 plant species to predict extinction risks
        \end{itemize}}
        \\
	\twentyitem
    	{2017 -}
		{Present}
        {NSF Postdoctoral Associate}
        {\href{https://www.kenyon.edu}{Kenyon College, OH, USA}}
        {}
        {
        {\begin{itemize}
        \item Explore and analyse the distribution of ca. 85000 plant species including approximately 9 million geographic points. 
        \item Research the consequences of habitat stability on current patterns of plant diversity using the distribution and phylogenetic relationships of ca. 24000 plant species. 
        \item Carry out paleohabitat reconstructions using machine learning methods
        \item Co-teaching of the BSc course ``Global Ecology and Biogeography'' 
        \item Teach and organise lectures and assignments for the Ecoinformatics course which include topics such as: 
        Introduction to data science R, data managing and processing, data visualization, spatial analysis and reproducibility \url{https://globalecologybiogeography.github.io/Ecoinformatics/}		
    \end{itemize}}
        }
    \\   
     \twentyitem
   		{2017}
		{}
        {Lecturer and demonstrator}
        {\href{http://www.nhm.ac.uk/}{The Natural History Museum, London, UK}}
        {}
        {
        {\begin{itemize}
        \item ``Phylogenetic approaches to studying diversification'' lecture for the Methods in Macroecology and Macroevolution course of the MSc Taxonomy and Biodiversity program
        \item Demostrator on the practical ``Fossils in phylogenetics'' for the Methods in Macroecology and Macroevolution course of the MSc Taxonomy and Biodiversity program
    \end{itemize}}
	}
    \\   
	\twentyitem
	{2013 - }
	{2017}
	{Demonstrator}
	{\href{https://www.imperial.ac.uk/}{Imperial College London, London, UK}}
	{}
	{
	 {\begin{itemize}
		\item Demonstrator on the Computational Biostatistics BSc course
		\item Demonstrator on the Ecology and Evolution BSc course
		\item Demonstrator on the Biodiversity and Conservation Biology BSc course
		\item Practicals included Introduction to R, Fundamentals of statistics in R, Phylogeny of mammals and pines, IUCN Red List, Biodiversity among lineages and over time, Delimiting species, Extinction risk patterns and correlates
		
	\end{itemize}}
    	}
        
%	%\twentyitem{<dates>}{<title>}{<location>}{<description>}
\end{twenty}

%----------------------------------------------------------------------------------------
%	 Publications
%----------------------------------------------------------------------------------------

\section{Selected publications}

2018 \textbf{Echeverr\'ia-Londo\~no S.,} Enquist, BJ, Neves, DM, Violle, C and Kerkhoff, AJ.  Plant functional diversity and the biogeography of biomes in North and South America, Frontiers in Ecology and Evolution, 6(DEC), 219.

2018 \textbf{Echeverr\'ia-Londo\~no S.,}  S{\"a}rkinen, T., Fenton, I. S., Knapp, S., \& Purvis, A. Dynamism and context dependency in the diversification of the megadiverse plant genus Solanum L.(Solanaceae). bioRxiv, 348961.

2016 \textbf{Echeverr\'ia-Londo\~no S.,} Newbold, T., Hudson, L. N., Hill, S. L., Contu, S., Lysenko, I., ... \& Purvis, A. Modelling and projecting the response of Colombian biodiversity to land-use change. Diversity and Distributions, 22: 1099-1111.

2015 Newbold, T., Hudson, L. N., Hill, S. L., Contu, S., Lysenko, I., Senior, R. A., Bennet D. J., Choimes A., Collen, B., Day, J., De Palma, A., Diaz, S., \textbf{Echeverr\'ia-Londo\~no S.,} ... \& Purvis, A. Global effects of land use on local terrestrial biodiversity. Nature, 520(7545), 45-50.

2011 \textbf{Echeverr\'ia-Londo\~no S., }\& Miranda-Esquivel, D. R. MartiTracks: A geometrical approach for identifying geographical patterns of distribution. PLoS ONE 6(4): e18460.


\end{document}